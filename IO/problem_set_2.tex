\documentclass{article}
\usepackage{amsmath} 
\usepackage{amsfonts}
\usepackage{booktabs}
\usepackage[a4paper, margin=2.5cm]{geometry}
\usepackage{float}   % for [H]
\usepackage{graphicx}   % for \includegraphics
\usepackage{tabularx}
\usepackage[utf8]{inputenc}
\usepackage{geometry}
\usepackage{booktabs}
\usepackage{longtable}
\usepackage{blindtext}
\usepackage{hyperref}
\usepackage[round]{natbib}
\usepackage{setspace}
\usepackage{array}
\usepackage{dcolumn}
\usepackage{threeparttable}
\usepackage{tikz}
\usepackage{amsmath}
\usetikzlibrary{decorations.pathreplacing}
\usepackage{pdflscape} % in your preamble
\usepackage{tabularray}
\setcounter{secnumdepth}{2}
\usepackage{amsthm}
\usepackage{pgfplots}
\pgfplotsset{compat=1.15}
\usepackage{mathrsfs}
\usetikzlibrary{arrows}
\definecolor{ccqqqq}{rgb}{0.8,0,0}
\definecolor{ududff}{rgb}{0.30196078431372547,0.30196078431372547,1}
\definecolor{xdxdff}{rgb}{0.49019607843137253,0.49019607843137253,1}
\setlength{\parskip}{0.45em}   % space between paragraphs
\setlength{\parindent}{0pt}    % optional: remove paragraph indentation

\setlength\parindent{0pt}

\begin{document}


\title{Part 2 IO problem set}
\author{Carlos Àlvarez, Jordi Torres}
\date{\today}

\maketitle


\section*{1.1}

\begin{itemize}

\item $(0,0)$
\[
\Pi_{im}^M \le 0
\iff
\hat\Pi_{im}^M + \epsilon_{im} \le 0
\iff
\epsilon_{im} \le -\hat\Pi_{im}^M,
\qquad \forall i\in\{1,2\}.
\]

\item $(1,1)$
\[
\hat\Pi_{im}^D + \epsilon_{im} \ge 0
\iff
\epsilon_{im} \ge -\hat\Pi_{im}^D,
\qquad \forall i\in\{1,2\}.
\]

\item Unique entry by firm $j$ occurs when:
\begin{enumerate}
    \item
    \[
    \hat\Pi_{jm}^M + \epsilon_{jm} \ge 0
    \;\cap\;
    \hat\Pi_{-jm}^M + \epsilon_{-jm} \le 0;
    \]
    \item
    \[
    \hat\Pi_{jm}^M + \epsilon_{jm} \ge 0
    \;\cap\;
    \hat\Pi_{-jm}^M + \epsilon_{-jm} \ge 0
    \;\cap\;
    \hat\Pi_{-jm}^D + \epsilon_{-jm} \le 0.
    \]
\end{enumerate}

\item The region of multiplicity is given by
\[
-\hat\Pi_{1m}^D \le \epsilon_{1m} \le -\hat\Pi_{1m}^M
\;\cap\;
-\hat\Pi_{2m}^D \le \epsilon_{2m} \le -\hat\Pi_{2m}^M,
\]
corresponding to Region~5 in Figure~\ref{fig:equilibrium_regions}.

\end{itemize}

Figure~\ref{fig:equilibrium_regions} illustrates the partition of the
$(\epsilon_1,\epsilon_2)$ space into the nine regions described above.

\section*{1.2}

The remaining probabilities can be written as follows.

The probability of no entry is
\[
\mathbf{P}(D_1=0,D_2=0)
=
\int_{-\infty}^{-\pi_1^M}
\int_{-\infty}^{-\pi_2^M}
\phi_2(u_1,u_2;\rho)\,du_1\,du_2
=
\Phi_2(-\pi_1^M,-\pi_2^M;\rho).
\]

Assume now that firm~1 moves first, so that equilibrium multiplicity is resolved
in favor of outcome $(1,0)$ (Region~4 in Figure~\ref{fig:equilibrium_regions}).
Then
\[
\mathbf{P}(1,0)
=
\int_{-\pi_1^M}^{-\pi_1^D}
\int_{-\infty}^{-\pi_2^D}
\phi_2(u_1,u_2;\rho)\,du_2\,du_1
=
\Phi_2(\infty,-\pi_2^D;\rho)
-
\Phi_2(-\pi_1^M,-\pi_2^D;\rho).
\]

Finally, $\mathbf{P}(0,1)$ is implied by the remaining probabilities.

\section*{1.3}

Table~\ref{tab:reg1} in the Appendix reports estimation results obtained under
alternative sign restrictions on the correlation parameter $\rho$. The
log-likelihood is lower when $\rho$ is restricted to be positive, indicating a
worse fit relative to the specification with negative correlation. Coefficient
estimates also differ substantially across the two cases. In particular, when
$\rho<0$, the estimated competition effect becomes statistically insignificant.
This is consistent with the interpretation that negative correlation in
unobserved shocks reflects greater heterogeneity between firms, reducing the
ability of the model to separately identify competitive effects from
idiosyncratic profitability differences.

Table~\ref{tab:reg2} reports estimates of $\rho$ when it is freely estimated. The
estimated value is approximately $0.3$, suggesting a moderate positive
correlation in unobserved shocks. The table also reports results obtained under
alternative assumptions about the identity of the firm that moves first. The
main differences across specifications arise in the estimated competition
effect. When firm~1 (Walmart) is assumed to enter first, the estimated impact of
competition on profits is smaller.

This pattern reflects the role of $\delta$ and $\rho$ in rationalizing observed
entry decisions. When Walmart is assumed to move first, the model attributes a
larger share of entry asymmetries to differences in unobserved profitability,
implying weaker competitive effects. Conversely, when the smaller firm is
assumed to move first, the model requires stronger competitive effects and more
similar shocks to rationalize observed outcomes.

Table~\ref{tab:reg3} confirms this interpretation. The estimated value of $\rho$
remains close to that obtained under the assumption that firm~2 moves first,
while the estimated effect of Walmart's entry on Kmart's profits increases. This
is consistent with Walmart being the larger firm, whose entry plausibly exerts a
stronger competitive pressure on Kmart.


\section*{1.4}

Table~\ref{tab:reg4} reports estimation results under fixed values of the
correlation parameter $\rho$ equal to $0.5$ and $1$. When shocks are highly
correlated, the model attributes a larger share of the observed variation in
entry outcomes to strategic interaction, leading to a higher estimated
competition effect. Conversely, when shocks are less correlated, heterogeneity
in unobserved profitability accounts for a greater fraction of entry decisions,
and the estimated competition effect decreases. The specification with
$\rho=1$ converges more quickly and attains a slightly lower log-likelihood,
reflecting the tighter structure imposed by perfectly correlated shocks.

Table~\ref{tab:reg5} reports results for the same model when $\rho$ is estimated
freely. In this specification, the competition effect and the correlation
parameter are not separately identified. The reason is that, in this simplified
setting, firms are symmetric in observables, and the only source of
cross-firm variation is the unobserved shock. Allowing $\rho$ to approach one
enables the model to rationalize the observed distribution of the number of
entrants across markets without requiring a distinct competition effect,
resulting in a flat likelihood in the $(\delta,\rho)$ direction.

This lack of identification also explains why the effects of $Z_m$ cannot be
estimated in this model. Since outcomes $(1,0)$ and $(0,1)$ are aggregated into
the same category $n=1$, there is no variation in the data that distinguishes
between the identities of entering firms. As a result, parameters that shift
relative profitability across firms are not identified.

Table~\ref{tab:reg6} presents results for a model with identical firms. This
specification is equivalent to the model in Part 1 under the restriction
$\rho=1$, since perfect correlation in a bivariate normal distribution implies
identical payoff shocks.

Finally, Table~\ref{tab:reg7} reports results from an ordered logit model for the number of entrants. This specification yields qualitatively similar results. By construction, firms are symmetric, and the ordered logit rationalizes observed
entry frequencies through threshold parameters of a latent profitability index.
These cutoffs capture market-level profitability and competitive pressures in a
reduced-form manner, analogous to the role played by the competition effect in
the structural model.

\section*{2}

 \subsection*{2.1}
 The game can be summarized in 2 equalities and 4 inequalities (or 8 inequalities). Let's start with the equalities: 

 \begin{enumerate}
    \item (1,1)  \(\rightarrow \mathbf{P}(\epsilon_{1}\geq -\pi_{1}^{D},\epsilon_{2}\geq -\pi_{2}^{D}, \rho )= \hat{P_{1,1}} \rightarrow M_1= \hat{P_{1,1}} -\mathbf{P}(\epsilon_{1}\geq -\pi_{1}^{D},\epsilon_{2}\geq -\pi_{2}^{D}, \rho )=0\)
    \item (0,0) \(\mathbf{P}(\epsilon_{1}\leq -\pi_{1}^{M},\epsilon_{2}\leq -\pi_{2}^{M}, \rho )= \hat{P_{0,0}} \rightarrow M_2= \hat{P_{0,0}}- \mathbf{P}(\epsilon_{1}\leq -\pi_{1}^{M},\epsilon_{2}\leq -\pi_{2}^{M}, \rho )=0\)

    

Now let's consider the problematic cases (1,0), (0,1). For these, we have already shown that $P_{0,1}$ or $P_{1,0}$ can not be point identified in the data -unless we make assumptions on who moves first etc (as done above) because there is a region of multiplicity where multiple equilibria can rationalize the data. 

However, these probabilities are not devoid of empirical content. Let's take (1,0). We know that the probability we estimate in the data $\hat{P_{0,1}}$ will be bounded between: 

$P_{0,1}^{l} \leq \hat{P_{0,1}} \leq P_{0,1}^{u}  $, where the lower bound is just the region where 0,1 is equilibrium not in multiplicity. The upper bound is the opposite: the region when we assume the selection function of the multiplicity area always selects this outcome. This gives us two inequalities: 

\item  $P_{0,1}^{l} \leq \hat{P_{0,1}} \rightarrow M_3= \hat{P_{0,1}}- P_{0,1}^{l} \geq 0 $
\item $\hat{P_{0,1}} \leq P_{0,1}^{u} \rightarrow M_4= P_{0,1}^{u} - \hat{P_{0,1}}  \geq 0$

Same exercise can be done with $P_{1,0}$. This leads with 8 moments, 2 equalities (4 inequalities) and 4 inequalities:

\item  $P_{1,0}^{l} \leq \hat{P_{1,0}} \rightarrow M_3= \hat{P_{1,0}}- P_{1,0}^{l} \geq 0 $
\item $\hat{P_{1,0}} \leq P_{1,0}^{u} \rightarrow M_4= P_{1,0}^{u} - \hat{P_{1,0}}  \geq 0$

\subsection*{2.2}
First, \(\hat{P_{1,1}},\hat{P_{0,0}} , \hat{P_{1,0}}, \hat{P_{0,1}}, \), can be directly estimated with the data. This can be done with nonparametric methods. (\textcolor{blue}{do we need to condition on X?}).

Second, we will estimate the rest using model implied probabilities - recycling some results from above, actually -all conditional to X. Let's go by parts: 

\begin{enumerate}
    \item \( p_{00}= \Phi_{2}(-\pi_{1}^{M}, -\pi_{2}^{M}, \rho)\)
    \item \( p_{11}= 1- \Phi_{2}(-\pi_{1}^{d}, \infty, \rho)- \Phi_{2}(-\pi_{1}^{d}, -\pi_{2}^{d}, \rho)- \Phi_{2}(\infty, -\pi_{2}^{d}, \rho) \)  
    \item For either $(1,0)$ or $(0,1)$ we have: \begin{itemize}
        \item \(P_{1,0}^{l}= \Phi_{2}(\infty, - \pi_{2}^{d}, \rho)-\Phi_{2}(- \pi_{1}^{m}, - \pi_{2}^{d}, \rho) \)
        \item  \(P_{1,0}^{u}= 1- \Phi_{2}(-\pi_{1}^{m}, -\pi_{2}^{d}, \rho)- p_{11} - p_{00}\) 
        \item By symmetry $(0,1)$ is just the same reversed. 
    \end{itemize} 
\end{enumerate}
 Where in this case \(\pi_{i}^{m}, \pi_{i}^{d}\) are profits under monopoly or duopoly as specified in the beginning for firm $i \in \{1,2\}$. 

 \subsection*{2.3}

 We can define the following objective function:

 \[
 Q(\theta)= \int h(M_{j}(\theta))_{+} dFx
 \]

 Where $j$ are the amount of moment inequalities we have and where $M$ designates the moment, which are functions of $\theta$.

 The inequalities are written in a positive way. In this sense, we want to penalize deviations for the implied inequalities in the model, so we will construct $Q(\theta)$ so that we only count deviations from our model\footnote{We could also write the two equalities as 4 inequalities and add them all up.}.

\end{enumerate}


 \subsection*{2.4}

\subsection*{2.4 Estimation of the identified set}

To estimate the identified set of structural parameters, we follow the
moment-inequality approach of \textbf{Ciliberto and Tamer (2009)} and
\textbf{Chernozhukov, Hong, and Tamer (2007)}. The procedure consists of the
following steps.

\begin{enumerate}

\item \textbf{Discretization of market characteristics.}  
Let $X^{c}$ denote the vector of continuous market characteristics.
We discretize the support of $X^{c}$ into two bins for each continuous variable
(dbenton, spc, population, urban). Within each bin, we replace the continuous
variable by its sample mean. This discretization induces a finite support for
$(X,Z)$ and allows for nonparametric estimation of conditional choice
probabilities.

\item \textbf{Nonparametric estimation of conditional probabilities.}  
For each combination $(x,z)$ in the discretized support of $(X,Z)$, we estimate
the conditional distribution of market outcomes nonparametrically:
\[
\hat P_n(Y = y \mid X = x, Z = z)
=
\frac{\sum_{i=1}^{n} \mathbf{1}\{Y_i = y, X_i = x, Z_i = z\}}
     {\sum_{i=1}^{n} \mathbf{1}\{X_i = x, Z_i = z\}},
\qquad y \in \{(1,1),(1,0),(0,1),(0,0)\}.
\]
Each distinct realization of $(x,z)$ defines a market type in the estimation
sample. For each market type, we therefore obtain four empirical conditional
probabilities $(\hat p_{11}, \hat p_{10}, \hat p_{01}, \hat p_{00})$.

\item \textbf{Construction of model-implied moments.}  
Let $\theta$ denote the vector of structural parameters, which includes
coefficients on market characteristics, firm-specific intercepts, the common
competition effect, and the correlation parameter of the payoff shocks.\footnote{
Specifically, $\theta$ includes coefficients on population, spc, urban, and
dbenton; regional indicators; a common competition parameter; firm-specific
intercepts; and the correlation parameter $\rho$.
}
For a given $\theta$:

\begin{enumerate}

\item We compute monopoly and duopoly profit indices,
$\pi_i^M(x,z;\theta)$ and $\pi_i^D(x,z;\theta)$, using the specified linear
profit functions.

\item Assuming jointly normal payoff shocks, we compute the model-implied
probabilities for each market type:
\[
p_{11}(\theta), \quad p_{00}(\theta), \quad
p_{10}^L(\theta), \quad p_{10}^U(\theta), \quad
p_{01}^L(\theta), \quad p_{01}^U(\theta),
\]
where $p_{10}^L(\theta)$ and $p_{01}^L(\theta)$ denote the lower bounds implied by
regions of unique equilibrium, and $p_{10}^U(\theta)$ and $p_{01}^U(\theta)$ denote
the corresponding upper bounds implied by equilibrium multiplicity.

\item We construct a vector of moment conditions $m(\theta \mid x,z)$ that
enforces equality restrictions for outcomes $(1,1)$ and $(0,0)$ and inequality
restrictions for outcomes $(1,0)$ and $(0,1)$.

\end{enumerate}

\item \textbf{Criterion function.}  
We define the sample criterion function as
\[
Q_n(\theta)
=
\int H\!\left(m(\theta \mid x,z)\right)_+ \, d\hat F_{X,Z}(x,z),
\]
where $H(\cdot)_+$ denotes a nonnegative loss function that penalizes violations
of the moment inequalities, and $\hat F_{X,Z}$ is the empirical distribution of
$(X,Z)$. The identified set is approximated by the collection of parameter values
$\theta$ for which $Q_n(\theta)$ is close to zero.

Given the size \(\theta\) we define the grid of parameters using a Uniform distribution for each parameter. We take previous models as reference, and define bounds on the uniform to include them. We run $2000$ simulations of $\theta$. In table \ref{fig:qtheta_hist} we can see the distribution of $Q(\theta)$. 

\textbf{Inference}
\textcolor{blue}{this I would leave if there is time left...It can be done, but it is more prioritary to finish the rest.}
We will use CHT to make inference\dots



\end{enumerate}



\section*{Appendix}



\begin{figure}[H]
\centering
\includegraphics[width=0.7\textwidth]{graph_areas.png}
\caption{Equilibrium regions.}
\label{fig:equilibrium_regions}
\end{figure}



\begin{table}[H]
\centering
\caption{Regression Results}
\label{tab:reg1}

\input{1_3.tex}

\end{table}

\begin{table}[H]
\centering
\caption{Regression Results}
\label{tab:reg2}

\input{1_3_2.tex}

\end{table}

\begin{table}[H]
\centering
\caption{Regression Results}
\label{tab:reg3}

\input{1_3_3.tex}

\end{table}

\begin{table}[H]
\centering
\caption{Regression Results}
\label{tab:reg4}

\input{1.4_1.tex}

\end{table}


\begin{table}[H]
\centering
\caption{Regression Results}
\label{tab:reg5}

\input{1.4_2.tex}

\end{table}


\begin{table}[H]
\centering
\caption{Regression Results}
\label{tab:reg6}

\input{1.4_3.tex}

\end{table}


\begin{table}[H]
\centering
\caption{Regression Results}
\label{tab:reg7}

\input{1.4_4.tex}

\end{table}


\begin{figure}[H]
\centering
\includegraphics[width=0.7\textwidth]{distribution_Q.png}
\caption{Distribution of the criterion function $Q(\theta)$.}
\label{fig:qtheta_hist}
\end{figure}



\end{document}