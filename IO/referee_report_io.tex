\documentclass{article}
\usepackage{amsmath} 
\usepackage{amsfonts}
\usepackage{booktabs}
\usepackage[a4paper, margin=2.5cm]{geometry}
\usepackage{float}   % for [H]
\usepackage{graphicx}   % for \includegraphics
\usepackage{tabularx}
\usepackage[utf8]{inputenc}
\usepackage{geometry}
\usepackage{booktabs}
\usepackage{longtable}
\usepackage{blindtext}
\usepackage{hyperref}
\usepackage{natbib} % <-- NEW: to handle references
\usepackage{setspace}
\usepackage{array}
\usepackage{dcolumn}
\usepackage{threeparttable}
\usepackage{tikz}
\usepackage{amsmath}
\usetikzlibrary{decorations.pathreplacing}
\usepackage{pdflscape} % in your preamble
\usepackage{tabularray}
\setcounter{secnumdepth}{2}
\usepackage{amsmath, amsthm}  % for math and theorem environments
\setlength{\parskip}{0.45em}   % space between paragraphs
\setlength{\parindent}{0pt}    % optional: remove paragraph indentation

\usepackage{titlesec}

\titlespacing*{\section}
{0pt}      % left margin
{1.0em}    % space before section
{0.5em}    % space after section


\begin{document}

\title{Referee Report: Welfare effects of Buyer and Seller Power (Demirer, Rubens)}
\author{Jordi Torres}
\date{\today}


\maketitle

\section*{1. Introduction}

In this paper, Demirer and Rubens provide the theoretical and empirical foundations to analyze the welfare effects of changes in buyer power in vertical structures when the type of vertical relation is unobserved by the econometrician or the policy maker. Welfare consequences of changes in buyer power crucially depend on the structure of the market: in a monopsonistic setting, increasing buyer market power can further distort the market, while under monopolistic vertical conduct increased buyer power may mitigate double marginalization. Given that in many markets the type of vertical conduct is not directly observed, the proposed framework is relevant for policy analysis.

The starting point is a model with increasing marginal costs upstream and downward-sloping demand downstream, under which equilibria exist under both monopsonistic and monopolistic vertical conduct. Most IO models rule out one of these conduct regimes by assumption in order to make conduct known. A key contribution of the paper is not to impose a specific type of vertical conduct, but to endogenize it within a unified framework by means of conduct-selection rules based on participation constraints.

The authors show that there exists an efficient level of buyer power at which the monopsonistic and monopolistic equilibria coincide with the efficient bargaining outcome. Crucially, this efficient level depends only on the relative elasticities of upstream costs and downstream demand and can therefore be recovered from these primitives. This efficient level acts as a threshold determining vertical conduct: under the proposed conduct-selection rules, comparing actual buyer power, $\beta$, to the efficient level, $\beta^{*}$, is sufficient to characterize the implied market structure. When $\beta > \beta^{*}$, the model predicts monopsonistic conduct; when $\beta < \beta^{*}$, monopolistic conduct prevails.

Finally, the authors apply the framework to several empirical settings. In some applications, actual buyer power cannot be estimated, but the efficient level $\beta^{*}$ can be inferred from elasticities, allowing the authors to assess whether buyer power is likely to be countervailing or distortionary. In their main empirical application to coal procurement in Texas, both $\beta$ and $\beta^{*}$ are estimated, which allows the authors to decompose observed welfare distortions into components attributable to buyer and seller power.

\section*{2. Model-Main results}


\section*{3. Limitation}


\section*{4. Conclusion}


 




\end{document}