\documentclass{article}
\usepackage{amsmath} 
\usepackage{amsfonts}
\usepackage{booktabs}
\usepackage[a4paper, margin=2.5cm]{geometry}
\usepackage{float}   % for [H]
\usepackage{graphicx}   % for \includegraphics
\usepackage{tabularx}
\usepackage[utf8]{inputenc}
\usepackage{geometry}
\usepackage{booktabs}
\usepackage{longtable}
\usepackage{blindtext}
\usepackage{hyperref}
\usepackage[round]{natbib}
\usepackage{setspace}
\usepackage{array}
\usepackage{dcolumn}
\usepackage{threeparttable}
\usepackage{tikz}
\usepackage{amsmath}
\usetikzlibrary{decorations.pathreplacing}
\usepackage{pdflscape} % in your preamble
\usepackage{tabularray}
\setcounter{secnumdepth}{2}
\usepackage{amsthm}
\newtheorem{definition}{Definition}



\setlength\parindent{0pt}

\begin{document}


\title{Replication Exercise -Set Identification, Mres}
\author{Jordi Torres}
\date{\today}


\maketitle

\section*{Exercise 1}

\begin{itemize}
    \item We can interpret $\beta_{l}$ as the increase in production given an increase in labor, keeping prices fixed. Derivation in section~\ref{ex1.1}
    \item Now we need to assume that both inputs are flexible and static. This is totally unrealistic for capital, as it is quite unlikely that farmers invest in capital non-dynamically (tools/machines last periods, 1920..old technology, even more) and that can quickly adjust their levels of capital in the short run. I derive results in section~\ref{ex1.1}
    \item 
    
\end{itemize}


\section{Appendix}

\subsection*{1}
\label{ex1.1}
Under perfect competition and one mobile input only (labor) we have that the firm solves: 

\[
\begin{aligned}
\min_{L_{f,t},\,K_{f,t} \ge 0} 
&\; L_{f,t} W_{f,t} + K_{f,t} R_{f,t} \\
\text{s.t.} 
&\; \bar{Q}_{f,t}
= L_{f,t}^{\beta_l} K_{f,t}^{\beta_k} \Omega_{f,t}
\end{aligned}
\]

From the first order condition on labor:
\[
W_{f,t}=\lambda \beta_{l} \left(\frac{Q_{f,t}}{L_{f,t}}\right)
\]
By definition: $\mu_{f,t}=\frac{p_{f,t}}{\lambda_{f,t}}$
\[
W_{f,t}=\frac{p_{f,t}}{\mu_{f,t}} \beta_{l} \left(\frac{Q_{f,t}}{L_{f,t}}\right)
\]

Under perfect competition $\mu_{f,t}$=1, then rearranging we get:
\[
\beta_{l}= \frac{W_{f,t}L_{f,t}}{P_{f,t}Q_{f,t}}
\]

Which we can estimate with a simple average. 

\textbf{Text B}

From the same problem above, if we take first order conditions with respect to labor and capital we get: 

\(
W_{f,t}L_{f,t}=\lambda \beta_{l}(Q_{f,t})
\)

\(
R_{f,t}K_{f,t}=\lambda \beta_{k}(Q_{f,t})
\)

If we sum the two expressions: \(
W_{f,t}L_{f,t}+R_{f,t}K_{f,t}=\lambda \beta_{l}(Q_{f,t}) + \lambda \beta_{k}(Q_{f,t})
\)

which reduces to:

\(
W_{f,t}L_{f,t}+R_{f,t}K_{f,t}=\lambda Q_{f,t} 
\)

Then if we take 1 or 2 and divide by the same in both sides: 

\(
\frac{W_{f,t}L_{f,t}}{W_{f,t}L_{f,t}+R_{f,t}K_{f,t}}=\frac{\lambda \beta_{l}(Q_{f,t})}{\lambda Q_{f,t}}
\)

This simplifies: \(
\frac{W_{f,t}L_{f,t}}{W_{f,t}L_{f,t}+R_{f,t}K_{f,t}}=\beta_{l}
\)

Same argument for the other. The markup also comes from the expression from above:

\(
\lambda= \frac{W_{f,t}L_{f,t}}{Q_{f,t}\beta_{l}}
\)
If we substitute $\beta_{l}$ into this expression and we rearrange we have: 

\(
\lambda=\frac{W_{f,t}L_{f,t}+R_{f,t}K_{f,t}}{Q_{f,t}}
\)

Thus:
\(
\mu=\frac{pQ_{f,t}}{W_{f,t}L_{f,t}+R_{f,t}K_{f,t}}
\)



\subsection*{2}


\end{document}