\documentclass{article}
\usepackage{amsmath} 
\usepackage{amsfonts}
\usepackage{booktabs}
\usepackage[a4paper, margin=2.5cm]{geometry}
\usepackage{float}   % for [H]
\usepackage{graphicx}   % for \includegraphics
\usepackage{tabularx}
\usepackage[utf8]{inputenc}
\usepackage{geometry}
\usepackage{booktabs}
\usepackage{longtable}
\usepackage{blindtext}
\usepackage{hyperref}
\usepackage{natbib} % <-- NEW: to handle references
\usepackage{setspace}
\usepackage{array}
\usepackage{dcolumn}
\usepackage{threeparttable}
\usepackage{tikz}
\usepackage{amsmath}
\usetikzlibrary{decorations.pathreplacing}
\usepackage{pdflscape} % in your preamble
\usepackage{tabularray}
\setcounter{secnumdepth}{2}

\setlength\parindent{0pt}




\begin{document}

\title{Exam- Quantitative Methods}
\author{Jordi Torres}
\date{\today}


\maketitle

\section*{Q1}

A stationary recursive competitive equilibrium consists of 
(i) value and policy functions $V(a,s,e)$, $c(a,s,e)$, $a'(a,s,e)$, 
(ii) factor prices $(r,w)$, 
(iii) aggregate quantities $(K,N)$, 
and (iv) a stationary distribution $\Gamma(a,s,e)$ over individual states, such that:

\begin{enumerate}
    \item \textbf{Households.} 
    Given prices $(r,w)$, the value function $V(a,s,e)$ solves
    \[
        V(a,s,e)
        = \max_{c,a'\ge 0} 
        \left\{ u(c) + \beta 
        \sum_{s'}\sum_{e'} P_s(s'|s) P_e(e'|e)\, V(a',s',e') \right\}
    \]
    subject to the budget constraint
    \[
        c + a' = w \,[e + b(1-e)]\, s\,(1-\tau_w) + (1+r)a,
    \]
    and $c \ge 0$. The associated policy functions are denoted $c(a,s,e)$ and $a'(a,s,e)$.

    \item \textbf{Firms.}
    Competitive firms have a standard Cobb--Douglas technology
    \[
        Y = z K^{\alpha} N^{1-\alpha}.
    \]
    Factor prices equal marginal products:
    \[
        r = \alpha z K^{\alpha-1} N^{1-\alpha} - \delta,
        \qquad
        w = (1-\alpha) z K^{\alpha} N^{-\alpha}.
    \]

    \item \textbf{Aggregation and market clearing.}
    The stationary distribution $\Gamma(a,s,e)$ over individual states is 
    consistent with the optimal policy $a'(a,s,e)$ and the exogenous transition 
    matrices $P_s$ and $P_e$.
    Given $\Gamma$, aggregate capital and effective labor are:
    \[
        K = \int a\, d\Gamma(a,s,e),
        \qquad
        N = \int e\, s \, d\Gamma(a,s,e).
    \]
    The capital market clears when the $K$ used by firms equals the $K$ implied by
    the distribution of assets:
    \[
        K = \int a\, d\Gamma(a,s,e).
    \]
    The labor market clears when the $N$ in the production function equals the
    aggregate effective labor supplied by employed workers:
    \[
        N = \int e\, s\, d\Gamma(a,s,e).
    \]
    Goods market clearing then holds by Walras' law.

    \item \textbf{Government budget.}
    The government finances unemployment benefits by a proportional tax on labor income.
    In the stationary equilibrium, its budget is balanced:
    \[
        \underbrace{\tau_w\, w \int [\,e + b(1-e)\,]\, s \, d\Gamma(a,s,e)}_{\text{labor income tax revenues}}
        \;=\;
        \underbrace{w\, b \int (1-e)\, s \, d\Gamma(a,s,e)}_{\text{UI payments}}.
    \]
\end{enumerate}






\end{document}