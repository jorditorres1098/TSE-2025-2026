\documentclass{article}
\usepackage{amsmath} 
\usepackage{amsfonts}
\usepackage{booktabs}
\usepackage[a4paper, margin=2.5cm]{geometry}
\usepackage{float}   % for [H]
\usepackage{graphicx}   % for \includegraphics
\usepackage{tabularx}
\usepackage[utf8]{inputenc}
\usepackage{geometry}
\usepackage{booktabs}
\usepackage{longtable}
\usepackage{blindtext}
\usepackage{hyperref}
\usepackage{natbib} % <-- NEW: to handle references
\usepackage{setspace}
\usepackage{array}
\usepackage{dcolumn}
\usepackage{threeparttable}
\usepackage{tikz}
\usepackage{amsmath}
\usetikzlibrary{decorations.pathreplacing}
\usepackage{pdflscape} % in your preamble
\usepackage{tabularray}
\setcounter{secnumdepth}{2}
\usepackage{amsmath, amsthm}  % for math and theorem environments

% Define a theorem environment
\newtheorem{theorem}{Theorem}

\setlength\parindent{0pt}




\begin{document}

\title{Takehome 3- DID}
\author{Jordi Torres}
\date{\today}


\maketitle

\section*{Exercise 1}

Let $(g,t)\in \{1, \dots, G\} \times \{1, \dots, T\} := N $ In our case, $g$ stands for municipalities and $t$ for term. Treatment in this case is defined as $d_{g,t}\in \{0,1\}$, where 1 indicates that the village was given the Seguro Popular at that time period. \\

The design was staggered because treatment was switched in only once $d_{g,t}=1$ and then it remains 1 (there is one exception of a switcher that is in treatment, switches off and then switcher back in, probably a data cleaning mistake. ). Intensity of treatment does not vary over time or change, no?

In table~\ref{tab:classification} we can see the proportion of villages that switch into or out of treatment and those that are stayers. For those that are stayers, I distinguish between the always treated and the always control. We can observe that most of the sample is always treated, while only 10\% of villages switch.

% latex table generated in R 4.4.0 by xtable 1.8-4 package
% Thu Nov  6 12:53:08 2025
\begin{table}[ht]
\centering
\begin{tabular}{lrr}
  \toprule
group & n & rel\_freq \\ 
  \midrule
Always Treated & 151 & 0.77 \\ 
  Never Treated &  26 & 0.13 \\ 
  Switcher &  18 & 0.09 \\ 
   \bottomrule
\end{tabular}
\caption{Classification of municipalities} 
\label{tab:classification}
\end{table}


We can write the following model: 

\[
Y_{g,t}= \sum_{g'=1}^{G}\gamma_{g'}\mathbf{1}\{g=g'\} + \sum_{t'=1}^{T}\gamma_{t'}\mathbf{1}\{t=t'\} + \beta_{fe}^{X} \mathbf{1}\{d_{g,t}=1\} + \textbf{X}'\beta+ \epsilon_{j,t}
\]
Where we cluster $\epsilon_{j,t}$ at the village level. 

We need to collapse the dataset at the level of analysis. I include the general case where we want to condition on a vector of characteristics at the g,t level to make sure assumptions below hold (as stated in Theorem 1 of Chaisemartin d'Haultefe DID)

\begin{theorem}
    In design CLA, if PT and NA hold, then: \(\mathbf{E}(\beta_{fe}^{X})=\text{ATT}\)
\end{theorem}


\textbf{Assumptions needed}

\begin{enumerate}
    \item \textbf{SUTVA (randomness of treatment)}: this implies that g's potential outcome depends only on g's treatment and not on other villages. This implies that there are no spillovers across villages of the Seguro Popular treatment (which seems reasonable).--> is this needed 100\%? Is this implied by any of the other two assumptions?
    \item \textbf{No anticipation (NA)}: current outcome does not depend on future realizations of the treatment. Here implies that agents don't expect the change in the subsidy and adapt their behavior? Initial condition also? what if villages were exposed to treatment before we start measuring them?
    \item \textbf{Parallel trend assumptions (PT)}: it implies that, in the absence of treatment, the trends that groups would have followed was the same. That is: for all $t\leq 2$ : \( \mathbf{E} (Y_{g,t}(0)-Y_{g,t-1}(0)) \), which means for untreated potential outcomes would have stayed the same. 
\end{enumerate}

Limitations of these assumptions? \textbf{\textcolor{blue}{ADD}} p.67 of their book, basically. 

\section*{Exercise 2}

In table~\ref{tab:fe_fd} I show the results of the TWFE estimator and the First difference estimator. Both coefficients are not significant; likely because of the low amount of switchers that we have in the dataset. Without forbidden comparisons. -->solution to this? with and without forbidden comparisons?


\begin{table}
\begin{center}
\begin{tabular}{l D{.}{.}{3.3} D{.}{.}{3.3} D{.}{.}{3.3} D{.}{.}{3.3}}
\toprule
 & \multicolumn{1}{c}{2WFE} & \multicolumn{1}{c}{2WFE} & \multicolumn{1}{c}{FD} & \multicolumn{1}{c}{FD} \\
\midrule
treat                   & -0.04  & -0.04  & -0.07  & -0.07  \\
                        & (0.07) & (0.07) & (0.09) & (0.09) \\
mean\_age               &        & -0.02  &        & -0.06  \\
                        &        & (0.06) &        & (0.05) \\
share\_formal           &        & 0.02   &        & 0.04   \\
                        &        & (0.06) &        & (0.07) \\
mean\_sec\_occup        &        & 0.12   &        &        \\
                        &        & (0.09) &        &        \\
(Intercept)             &        &        & -0.00  & 0.01   \\
                        &        &        & (0.02) & (0.02) \\
\midrule
Num. obs.               & 772    & 772    & 577    & 577    \\
Num. groups: cvemun     & 195    & 195    &        &        \\
Num. groups: quarter    & 4      & 4      &        &        \\
R$^2$ (full model)      & 0.72   & 0.72   &        &        \\
R$^2$ (proj model)      & 0.00   & 0.00   &        &        \\
Adj. R$^2$ (full model) & 0.63   & 0.63   &        &        \\
Adj. R$^2$ (proj model) & -0.00  & -0.00  &        &        \\
R$^2$                   &        &        & 0.00   & 0.00   \\
Adj. R$^2$              &        &        & -0.00  & -0.00  \\
\bottomrule
\multicolumn{5}{l}{\scriptsize{$^{***}p<0.01$; $^{**}p<0.05$; $^{*}p<0.1$}}
\end{tabular}
\caption{Fixed Effects and First Differences Models}
\label{tab:fe_fd}
\end{center}
\end{table}
 


\textcolor{blue}{textbf{Note: revise this ugly ass table}} Solved a way to put latex tables directly from R to Latex but with the problem of 1. ugly as fuck, 2. unable to put it where I want to. 


Here to revise a little bit the implementation of the first difference estimator etcetera in R and modify a little bit so the table is correct. 




\section*{Exercise 5}

\[
\mathbf{E}(w_{j})= U , \forall j\in \{f,i\}
\]

\[
\frac{w_{f}^{*}+ B_{f} + \eta_{f}U}{\rho + \eta_{f}}=U
\]

\[
w_{f}^{*}= U\rho - Bj
\]

Thus, although the total compensation will the be the same across informal and formal jobs (i.e $U\rho$), if \(B_f>B_i \iff w_f>w_i\). Higher amenities in terms of social security conditions in formal jobs will decrease the reservation wage. (no differential in skill-type of contract/informality + assumes workers with different skills have the same bargaining power.)

\section*{Exercise 6}
I will first try to simplify the expression of \(w_j^{*}\), given that this is an essential equilibrium element in the likelihood. This follows the procedure we did in class. 

We use the results of the previous exercise and the definition given in the exercise to write:

\[
w_j^{*}= b - B_{j} + \sum_{j} \lambda_{j} 
\left( 
\int_{w_{j}<w_{j}^{*}} U\, dF_j(w_j) 
+ 
\int_{w_j\geq w_{j}^{*}} E_j(w_j)\, dF_j(w_j) 
- U 
\right)
\]

Collecting terms and noting that 
\(\int_{w_{j}<w_{j}^{*}} dF_j(w_j) = F_j(w_j^{*})\), we obtain:

\[
w_j^{*}= b - B_{j} + \sum_{j} \lambda_{j} 
\left( 
U F_j(w_{j}^{*}) - U 
+ 
\int_{w_j\geq w_{j}^{*}} E_j(w_j)\, dF_j(w_j) 
\right)
\]

This simplifies to:

\[
w_j^{*}= b - B_{j} + \sum_{j} \lambda_{j} 
\left( 
- U \big(1 - F_j(w_{j}^{*})\big) 
+ 
\int_{w_j\geq w_{j}^{*}} E_j(w_j)\, dF_j(w_j) 
\right).
\]

Adding and subtracting \(U\) inside the integral and simplifying the 
\(U\big(1-F_j(w_j^*)\big)\) terms gives:

\[
w_j^{*}= b - B_{j} + \sum_{j} \lambda_{j} 
\int_{w_j\geq w_{j}^{*}} 
\big( 
E_j(w_j) - U 
\big)\, dF_j(w_j)
\]

We then substitute for the definition of \(E_j(w_j)\):

\[
E_j(w_j) = \frac{w_j + B_j + \eta_j U}{\rho + \eta_j}
\]

Using \(E_j(w_j^*)=U\) implies \(E_j(w_j)-U=\dfrac{w_j-w_j^*}{\rho+\eta_j}\)Replacing this into the expression above, we obtain:

\[
w_j^{*}= b - B_{j} + \sum_{j} 
\frac{\lambda_{j}}{\rho + \eta_j} 
\int_{w_j\geq w_{j}^{*}} 
\big(w_j - w_{j}^{*}\big)\, dF_j(w_j),
\]

which holds for each \(j \in \{i,f\}\)

\vspace{1em}

Then we construct the likelihood contributions of employed and unemployed people in the following way.

\paragraph{Unemployed.} Let \(\tilde F_j(w_j^*) \equiv 1 - F_j(w_j^*)\) and define the sector-specific job-finding hazards \(a_j \equiv \lambda_j \tilde F_j(w_j^*)\). The total exit hazard from unemployment is
\[
h_u \;=\; a_i + a_f \;=\; \lambda_i \tilde F_i(w_i^*) + \lambda_f \tilde F_f(w_f^*)
\]
The likelihood of an ongoing unemployment spell of duration \(t_u\) is
\[
L(t_u, u)
\;=\;
\underbrace{h_u \exp(-h_u t_u)}_{\text{duration density}}
\times 
\underbrace{p(u)}_{\text{steady-state mass at }U}\!,
\]
where, for the three-state system \((U,E_i,E_f)\),
\[
p(u) \;=\; \Big( 1 + \tfrac{a_i}{\eta_i} + \tfrac{a_f}{\eta_f} \Big)^{-1}
\]

\paragraph{Employed.} For employed individuals with observed sector \(j\) and wage \(w\),
\[
L(w,t_e, e_j)
\;=\;
\underbrace{\frac{f_j(w)}{\tilde F_j(w_j^*)}}_{\text{accepted-wage density on }[w_j^*,\infty)}
\;\times\;
\underbrace{\eta_j e^{-\eta_j t_e}}_{\text{job-duration density}}
\;\times\;
\underbrace{p(e_j)}_{\text{steady-state mass at }E_j},
\qquad
p(e_j)
\;=\;
\frac{a_j/\eta_j}{\,1 + \frac{a_i}{\eta_i} + \frac{a_f}{\eta_f}\,}.
\]


\textbf{Note:} we are assuming steady state, which implies \(a_i*\lambda_i=u*\eta_i\). This gives the expression of the probabilities from above

Thus, the overall likelihood function can be written as:

\[
L(\theta)= 
\prod_{i \in U} L(t_{u,i}, u) 
\times 
\prod_{i \in E} L(w_i, e) .
\]


And if we take logs the previous expression simply becomes:

\[
LL(\theta)
=
\sum_{i\in U}
\Big(
\log(h_u)
- h_u\, t_{u,i}
+ \log p(u)
\Big)
\mathbf{1}\{i\in U\}
\;+\;
\sum_{i\in e_j}
\Big(
\log(f_j(w_i))
- \log\big(\tilde F_j(w_j^{*})\big)
+ \log(\eta_j)
- \eta_j t_{e,i}
+ \log p(e_j)
\Big)
\mathbf{1}\{i\in e_j\}.
\]




\textbf{Identification intuition.}

The key reduced–form object that we observe in unemployment data is the
sector–specific job–finding probabilty:
\[
a_j \;\equiv\; \lambda_j \tilde F_j(w_j^*), 
\qquad 
\tilde F_j(w_j^*) = 1 - F_j(w_j^*).
\]
This $a_j$ is the probability that a worker receives and accepts an offer from sector $j$. From unemployment durations we can identify
$a_j$, but we cannot separate $\lambda_j$ from the acceptance probability
$\tilde F_j(w_j^*)$. \\

Job durations identify $\eta_j$ directly, because
\[
f(t_e\mid e_j) = \eta_j e^{-\eta_j t_e}
\]
is an exponential density. Hence $\eta_j$ is point–identified. \\

The accepted wage distribution identifies only the truncated density
\[
f_j(w \mid w \ge w_j^*) 
= 
\frac{f_j(w;\mu_j,\sigma_j)}{1-F_j(w_j^*;\mu_j,\sigma_j)},
\]
so we learn the shape parameters $(\mu_j,\sigma_j)$ of the offer distribution,
but we do not learn the location of the truncation point $w_j^*$ itself.
Because rejected offers are never observed, the data cannot recover down $w_j^*$ or
$\tilde F_j(w_j^*)=1-F_j(w_j^*)$. \\

Putting these facts together, all the objects in the likelihood can be written
as functions of
\[
a_i,\,a_f,\qquad
\eta_i,\,\eta_f,\qquad
(\mu_j,\sigma_j),
\]
and the \textbf{reservation wage $w_j^*$ always enters through the acceptance
probability $\tilde F_j(w_j^*)$ and therefore through $a_j$}.

Hence the likelihood can be rewritten as
\[
LL(\theta)
=
\tilde L(a_i,a_f,\eta_i,\eta_f,\mu_i,\mu_f,\sigma_i,\sigma_f),
\]
and $\lambda_j$ never appears separately, only inside $a_j$.

if we use the chain rule-\dots
\[
\frac{\partial LL}{\partial \lambda_j}
=
\frac{\partial \tilde L}{\partial a_j}
\frac{\partial a_j}{\partial \lambda_j}.
\]
Since $a_j = \lambda_j (1-F_j(w_j^*))$, different $(\lambda_j,w_j^*,\mu_j,\sigma_j)$
combinations that keep $a_j$ and the truncated density unchanged produce the
same likelihood.

\medskip

\textit{Intuition:} The data identify $a_j$ and $\eta_j$ separately, but they do
not identify $\lambda_j$ and the reservation wage $w_j^*$ separately. The
acceptance probability $\tilde F_j(w_j^*)$ and the offer arrival rate $\lambda_j$ enter the likelihood only through their product $a_j$, so the lower tail of the offer distribution and $\lambda_j$ are not separately identified. This is why fixing w* helps recovering the \(\lambda_j\) parameters. 


\section*{Exercise 7}

I first fixed the wages to be the minimum for both informal and formal jobs. Then I am able to identify \(\lambda_j\). Note that steady state imposes that \(\lambda_i=\lambda_f\). The code can be found in the jupyter notebook attached (implemented in Julia).

These are the results: 

\begin{table}[htbp]
\centering
\caption{Structural Parameter Estimates}
\begin{tabular}{lc}
\hline\hline
\textbf{Parameter} & \textbf{Estimate} \\
\hline
$\mu_i$      & 2.88049 \\
             & (0.00709) \\[0.25em]

$\mu_f$      & 3.13613 \\
             & (0.00609) \\[0.25em]

$\sigma_i$   & 0.57956 \\
             & (0.00885) \\[0.25em]

$\sigma_f$   & 0.52337 \\
             & (0.00868) \\[0.25em]

$\lambda_i$  & 0.13317 \\
             & (0.03409) \\[0.25em]

$\lambda_f$  & 0.15634 \\
             & (0.03350) \\[0.25em]

$\eta_i$     & 0.01007 \\
             & (0.01209) \\[0.25em]

$\eta_f$     & 0.01036 \\
             & (0.01132) \\
\hline
\hline
\multicolumn{2}{l}{\footnotesize Standard errors in parentheses.} \\
\multicolumn{2}{l}{\footnotesize All parameters except $\mu_i,\mu_f$ reported in exponentiated form.} \\
\end{tabular}
\end{table}


From the parameters estimates, I am able to recover the value of unemployed for informal and formal jobs (which should be the same by imposition?):

\begin{table}[htbp]
\centering
\caption{Recovered Non–Wage Utility Parameters}
\begin{tabular}{lc}
\hline\hline
\textbf{Parameter} & \textbf{Value} \\
\hline
$b_i$ & 35.246 \\[0.25em]
$b_f$ & 44.563 \\
\hline\hline
\end{tabular}
\end{table}

Standard errors can be computed with Delta method? Left due to time constraints. 


\section*{Exercise 8}

For this exercise I will use the parameters we got from the previous exercise to simulate wage profiles of the original sample (the one we used for 1-4). The process I follow is the following: 

\begin{enumerate}
    \item I recover the $w_j^{*}$ under the newly defined Bi. Given that I have been able to estimate b in the previous exercise, I can recover the reservation wage for the new $B_{f}$.
    \item Then, I simply estimate the truncated cdf's using this new reservation wages and the parameters \(\mu_j, \sigma^{2}_{j}\) that I got from the model. 
    \item I simulate wages depending on the type of occupation that individuals have in the original dataset in each quarter (and depending on their treatment status; if $sp=1$, formal get wages from different cdf\dots). If unemployed, wages are 0, if employed in formal then I use cdf formal, if informal I use the other (and account for $sp=1$, $sp=0$).
    \item This gives a simulated wage for each individual. Then I apply TWFE as done in exercise 2. 
\end{enumerate}


These are the results: 

\begin{table}[H]
\centering
\caption{TWFE Estimate Using Simulated Counterfactual Wages}
\begin{tabular}{lccc}
\hline\hline
 & Estimate & Std. Error & p-value \\
\hline
SP (treated) & 0.0488 & (0.0554) & 0.379 \\[0.3em]
\hline
Observations & \multicolumn{3}{c}{5284} \\
R-squared    & \multicolumn{3}{c}{0.037} \\
\hline\hline
\end{tabular}
\begin{flushleft}
\footnotesize \textit{Notes:} Standard errors in parentheses.  
Dependent variable is simulated log hourly wage.  
Estimation includes individual and quarter fixed effects.  
\end{flushleft}
\end{table}


\textbf{Revise this, not clear}

\section{Appendix}

\textbf{Derivatives}

We want to argue that some parameters can't be identified. So we take derivatives with respect to the four primitives of the model. Focusing on how the hazard $h_u$ and the truncation term $\tilde F_j(w_j^*)$ enter the likelihood, we can write (ignoring mixture issues and keeping $N_u$ and $N_e$ as the numbers of unemployed and employed observations, respectively):

\[
\frac{\partial LL(\theta)}{\partial \lambda_{j}}
=
N_{u}
\left[
\left(\frac{1}{h_{u}} - \bar t_{u}\right)
\frac{\partial h_{u}}{\partial \lambda_{j}}
\right]
-
N_{e}
\left[
\frac{1}{\tilde{F}_{j}(w_{j}^{*})}
\frac{\partial \tilde{F}_{j}(w_{j}^{*})}{\partial \lambda_{j}}
\right],
\]
where $\bar t_u$ is the average unemployment duration.

If we expand the expressions (ignoring cross–sector effects on the other reservation wage), then:
\[
\begin{aligned}
\frac{\partial h_u}{\partial \lambda_j}
&=
\tilde F_j(w_j^{*})
+
\lambda_j \frac{\partial \tilde F_j(w_j^{*})}{\partial \lambda_j}
=
\tilde F_j(w_j^{*})
-
\lambda_j f_j(w_j^{*}) \frac{\partial w_j^{*}}{\partial \lambda_j},
\\[0.5em]
\frac{\partial \tilde F_j(w_j^{*})}{\partial \lambda_j}
&=
\frac{\partial \tilde F_j(w_j^{*})}{\partial w_j^{*}}
\frac{\partial w_j^{*}}{\partial \lambda_j}
=
- f_j(w_j^{*}) \frac{\partial w_j^{*}}{\partial \lambda_j},
\end{aligned}
\]
where $f_j(\cdot)$ is the density of $F_j(\cdot)$.

Thus:
\[
\begin{aligned}
\frac{\partial LL(\theta)}{\partial \lambda_{j}}
&=
N_{u}
\left[
\left(
\frac{1}{h_{u}} - \bar t_{u}
\right)
\left(
\tilde{F}_{j}(w_{j}^{*})
-
\lambda_{j} f_{j}(w_{j}^{*})
\frac{\partial w_{j}^{*}}{\partial \lambda_{j}}
\right)
\right]
-
N_{e}
\left[
\frac{1}{\tilde{F}_{j}(w_{j}^{*})}
\left(
- f_{j}(w_{j}^{*})
\frac{\partial w_{j}^{*}}{\partial \lambda_{j}}
\right)
\right]
\\[0.5em]
&=
N_{u}
\left[
\left(
\frac{1}{h_{u}} - \bar t_{u}
\right)
\left(
\tilde{F}_{j}(w_{j}^{*})
-
\lambda_{j} f_{j}(w_{j}^{*})
\frac{\partial w_{j}^{*}}{\partial \lambda_{j}}
\right)
\right]
+
N_{e}
\left[
\frac{f_{j}(w_{j}^{*})}{\tilde{F}_{j}(w_{j}^{*})}
\frac{\partial w_{j}^{*}}{\partial \lambda_{j}}
\right].
\end{aligned}
\]

Hence, $\lambda_j$ enters both directly (through $\tilde F_j(w_j^*)$ in $h_u$) and indirectly through the equilibrium object $w_j^{*}$.

Then, if we take derivatives with respect to $\eta_j$ we can write:
\[
\frac{\partial LL(\theta)}{\partial \eta_j}
= 
N_u\left[
\left(\frac{1}{h_u} - \bar t_{u}\right)\frac{\partial h_u}{\partial \eta_j}
\;+\;
\frac{1}{p_u}\frac{\partial p_u}{\partial \eta_j}
\right]
-
N_e\left[
\frac{1}{\tilde{F}_{j}(w_j^*)}\frac{\partial \tilde{F}_{j}(w_j^*)}{\partial \eta_j}
\;+\;
\frac{1}{p_e}\frac{\partial p_e}{\partial \eta_j}
+
\frac{1}{\eta_i} -t_{e,i}
\right].
\]

Doing the same type of expansion as before:
\[
\begin{aligned}
\frac{\partial h_u}{\partial \eta_j}
&=
\lambda_j \frac{\partial \tilde F_j(w_j^{*})}{\partial \eta_j}
+
\lambda_i \frac{\partial \tilde F_i(w_i^{*})}{\partial \eta_j}
\\
&=
-\lambda_j f_j(w_j^{*}) \frac{\partial w_j^{*}}{\partial \eta_j}
-
\lambda_i f_i(w_i^{*}) \frac{\partial w_i^{*}}{\partial \eta_j},
\\[0.5em]
\frac{\partial \tilde{F}_j(w_j^{*})}{\partial \eta_j}
&=
- f_j(w_j^{*}) \frac{\partial w_j^{*}}{\partial \eta_j},
\end{aligned}
\]
so thatt
\[
\begin{aligned}
\frac{\partial LL(\theta)}{\partial \eta_j}
=
N_u\Bigg[
&\left(
\frac{1}{h_u} - \bar t_{u}
\right)
\left(
-\lambda_j f_j(w_j^{*}) \frac{\partial w_j^{*}}{\partial \eta_j}
- \lambda_i f_i(w_i^{*}) \frac{\partial w_i^{*}}{\partial \eta_j}
\right)
+
\frac{1}{p_u}\frac{\partial p_u}{\partial \eta_j}
\Bigg]
\\
&
- N_e\Bigg[
\frac{1}{\tilde{F}_j(w_j^{*})}
\left(
- f_j(w_j^{*}) \frac{\partial w_j^{*}}{\partial \eta_j}
\right)
+
\frac{1}{p_e}\frac{\partial p_e}{\partial \eta_j} +
\frac{1}{\eta_i} -t_{e,i}
\Bigg]
\\[0.5em]
=
N_u\Bigg[
&\left(
\frac{1}{h_u} - \bar t_{u}
\right)
\left(
-\lambda_j f_j(w_j^{*}) \frac{\partial w_j^{*}}{\partial \eta_j}
- \lambda_i f_i(w_i^{*}) \frac{\partial w_i^{*}}{\partial \eta_j}
\right)
+
\frac{1}{p_u}\frac{\partial p_u}{\partial \eta_j}
\Bigg]
\\
&+
N_e\Bigg[
\frac{f_j(w_j^{*})}{\tilde{F}_j(w_j^{*})}
\frac{\partial w_j^{*}}{\partial \eta_j}
-
\frac{1}{p_e}\frac{\partial p_e}{\partial \eta_j} +
\frac{1}{\eta_i} -t_{e,i}
\Bigg].
\end{aligned}
\]

Same thing happens here: $\eta_j$ affects the likelihood only through equilibrium objects such as $w_j^{*}$ and the steady-state probabilities $p_u$ and $p_e$.

Finally, for the parameters of the wage offer distribution we can schematically write (for sector $j$):
\[
\frac{\partial LL(\theta)}{\partial \mu_j}
= 
\kappa \,\frac{(\ln w_j - \mu_j)}{\sigma_{j}^{2}}
+ \frac{\partial \Phi(\cdot)}{\partial \mu_j},
\]
and for the variance parameter:
\[
\frac{\partial LL(\theta)}{\partial \sigma^{2}_{j}}
= 
\kappa \,\frac{(\ln w_j - \mu_j)^{2}}{2\sigma_{j}^{4}}
+ \frac{\partial \Phi(\cdot)}{\partial \sigma^{2}_{j}},
\]
where $\kappa$ collects constants and sample size terms, and $\Phi(\cdot)$ denotes the contribution of the truncation and duration components of the likelihood. The data only allow us to identify certain combinations of the structural parameters. For instance, $a_j = \lambda_j \tilde{F}_j(w_j^{*})$ can be identified, but it is a composite of $\lambda_j$ and the reservation wage $w_j^{*}$ (and thus of $\mu_j,\sigma_j$). The same argument applies to $\eta_j$.

\end{document}