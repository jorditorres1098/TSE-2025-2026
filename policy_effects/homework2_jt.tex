\documentclass{article}
\usepackage{amsmath} 
\usepackage{amsfonts}
\usepackage{booktabs}
\usepackage[a4paper, margin=2.5cm]{geometry}
\usepackage{float}   % for [H]
\usepackage{graphicx}   % for \includegraphics
\usepackage{tabularx}
\usepackage[utf8]{inputenc}
\usepackage{geometry}
\usepackage{booktabs}
\usepackage{longtable}
\usepackage{blindtext}
\usepackage{hyperref}
\usepackage{natbib} % <-- NEW: to handle references
\usepackage{setspace}
\usepackage{array}
\usepackage{dcolumn}
\usepackage{threeparttable}
\usepackage{tikz}
\usepackage{amsmath}
\usetikzlibrary{decorations.pathreplacing}
\usepackage{pdflscape} % in your preamble
\usepackage{tabularray}
\setcounter{secnumdepth}{2}

\setlength\parindent{0pt}




\begin{document}

\title{Homework 2}
\author{Jordi Torres}
\date{\today}


\maketitle




\section{Question 1}

Following Cattaneo et al., I select windows around each cutoff where covariates are balanced at the 15 \% level (i ALSO use the test of diff in means to be as less restrictive as possible). For the population running variable, no such window exists (recommended [0; 0]); the local-randomization assumption therefore fails.For the time running variable, a balanced window of ± 1.6 units is obtained with 147 observations (24 below, 123 above).

Within this window, as shown in table \ref{table1_localrda}, randomization-based inference yields a significant wage discontinuity of 326 units $(p < 0.01)$, while the teacher competency score shows no significant change $(p \approx 0.98)$.

This holds only if the following assumptions hold:

\begin{enumerate}
    \item As if random treatment assignment->covariates test
    \item SUTVA->argue this with the same way they do it in the paper, cite them.SUTVA may be violated if the policy triggered spillovers through teacher sorting around  the population cutoff – e.g. if teachers who chose a position in a high-bonus school just  below the threshold would have otherwise chosen a position just above the threshold in  the absence of the wage bonus policy.
    \item No manipulation at the cutoff->rdsensitivity
\end{enumerate}


\begin{table}[H]
\centering
\caption{Local Randomization RD Estimates}
\begin{tabular}{lcccccc}
\toprule
\textbf{Running Variable} & \textbf{Outcome} & \textbf{Recommended Window} & \textbf{Obs. Below} & \textbf{Obs. Above} & \textbf{Effect} & \textbf{p-value} \\
\midrule
Population & Wage  & [0 ; 0]             & 0   & 20  & --      & --     \\
Time       & Wage  & [-1.61 ; 1.61]      & 24  & 123 & 326.43  & 0.000  \\
Population & Score & [0 ; 0]             & 0   & 20  & --      & --     \\
Time       & Score & [-1.61 ; 1.61]      & 24  & 123 & -0.08   & 0.986  \\
\bottomrule
\end{tabular}
\label{table1_localrd}
\end{table}

In table XX I show some preliminary evidence of graphical RD around the cutoff. I have restricted the dataset to be around the cutoff and then I have grouped the dataset in similar bins -not using the Cattaneo et al approach- but very similarly. Just to provide some evidence. I have also fitted a kernel. We can observe that just looking at the graph there seems not to be a dis

Figure XX presents the randomization sensitivity analysis for the four combinations of running and outcome variables. I will consider only the time variable, as the balance test already did not support local effects of population. For this variable the wage effect is positive and significant in narrow windows, remaining robust as the window widens—consistent with a credible local randomization design. The score outcome shows small and insignificant effects, indicating no impact on academic performance.
\textbf{Revise this, unclear}

Finally, in tables XX and XX I show the results of the min-pvalue. Here we observe that clearly population does not capture a local effect, as the p-values of balancedness only breach 0.15 with larger bandwidths, while the opposite is true for the time variable, where the effect seems more locally robust. 



\section{Question 2}

\subsection{Continuity parametric}
\begin{equation}
Y_{is} = \alpha + \beta_1 X_{is} + \tau D_{is} + \beta_2 (D_{is} \times X_{is}) + \varepsilon_{is},
\label{eq:global_rd}
\end{equation}


\begin{table}[H]
\centering
\caption{Continuity-based (Global Parametric) RD Estimates}
\label{tab:global_rd}
\begin{tabular}{lcccccc}
\hline
Outcome & Running Variable & RD Estimate ($\hat{\tau}$) & Std. Error & t-value & p-value & Significance \\
\hline
Wage   & Population & 284.32 & 9.78 & 29.08 & $<0.001$ & *** \\
Score  & Population & 4.49   & 1.08 & 4.17  & $<0.001$ & *** \\
Wage   & Time       & 324.21 & 8.48 & 38.25 & $<0.001$ & *** \\
Score  & Time       & -1.52  & 8.48 & -0.18 & 0.858 &  \\
\hline
\multicolumn{7}{l}{\footnotesize Notes: Clustered standard errors at the school level.} \\
\multicolumn{7}{l}{\footnotesize Specification: $Y = \alpha + \beta_1 X + \tau D + \beta_2 D\times X$. The coefficient $\tau$ measures the RD discontinuity at the cutoff.}\\
\end{tabular}
\end{table}



\subsection{Continuity non-parametric}


\begin{table}[H]
\centering
\caption{Continuity-based (Local Non-Parametric) RD Estimates using \texttt{rdrobust}}
\label{tab:local_rd}
\begin{tabular}{lccccccc}
\hline
Outcome & Running Var & RD Estimate ($\hat{\tau}$) & Std. Error & p-value & Bandwidth ($h$) & $N_L$ / $N_R$ & Significance \\
\hline
Wage   & Population & -205.18 & 23.45 & $<0.001$ & 161.5 & 10,130 / 4,658 & *** \\
Wage   & Time       & 347.15  & 12.85 & $<0.001$ & 32.8  & 6,151 / 8,637  & *** \\
Score  & Population & -8.19   & 2.07  & $<0.001$ & 211.0 & 10,130 / 4,658 & *** \\
Score  & Time       & 4.88    & 1.82  & 0.007 & 27.8 & 6,151 / 8,637 & ** \\
\hline
\multicolumn{8}{l}{\footnotesize Notes: Estimates computed using \texttt{rdrobust} with triangular kernel and MSE-optimal bandwidths.}\\
\multicolumn{8}{l}{\footnotesize $Y_{is} = \alpha + \tau D_{is} + f_-(X_{is}) + f_+(X_{is}) + \varepsilon_{is}$, where $\tau$ measures the discontinuity at the cutoff.}\\
\multicolumn{8}{l}{\footnotesize Cluster-robust standard errors at the school level. Significance levels: *** $p<0.01$, ** $p<0.05$, * $p<0.1$.}\\
\end{tabular}
\end{table}


\textbf{add also graphs}



\subsection{Assumptions and test}

I think I may need to add stuff on the sensitivity? rd sensitivity. 

No manipulation (rddensity) -->this is key, but do it more systematic, review this once I am done with this bullshit.

Covariates smooth at the cutoff

Placebo cutoffs (optional): run RD at fake cutoffs (e.g., ±20 units) to show no spurious jumps.

Robustness:

\begin{table}[H]
\centering
\caption{Robustness of Local RD Estimates to Polynomial Order and Kernel Choice}
\label{tab:rd_robustness}
\begin{tabular}{llcccccc}
\hline
Specification & Outcome & Running Var & $\hat{\tau}$ & SE & p-value & Bandwidth ($h$) & Obs (L/R) \\
\hline
\textbf{p=2, Triangular} & Wage & Population & -227.06 & 37.67 & $<0.001$ & 159.9 & 10,130 / 4,658 \\
                         & Wage & Time       & 335.92  & 18.92 & $<0.001$ & 29.1  & 6,151 / 8,637 \\
                         & Score & Population & -7.38  & 2.84 & 0.009 & 215.2 & 10,130 / 4,658 \\
                         & Score & Time       & 1.27   & 2.87 & 0.658 & 24.7 & 6,151 / 8,637 \\
\textbf{p=3, Triangular} & Wage & Population & -228.46 & 39.30 & $<0.001$ & 252.6 & 10,130 / 4,658 \\
                         & Wage & Time       & 333.28  & 21.08 & $<0.001$ & 40.5  & 6,151 / 8,637 \\
                         & Score & Population & -6.14  & 3.39 & 0.070 & 231.9 & 10,130 / 4,658 \\
                         & Score & Time       & 0.82   & 3.14 & 0.794 & 36.3 & 6,151 / 8,637 \\
\textbf{p=1, Epanechnikov} & Wage & Population & -201.69 & 22.14 & $<0.001$ & 161.9 & 10,130 / 4,658 \\
                           & Wage & Time       & 352.15  & 13.43 & $<0.001$ & 28.2  & 6,151 / 8,637 \\
                           & Score & Population & -8.04  & 2.20 & 0.0002 & 176.4 & 10,130 / 4,658 \\
                           & Score & Time       & 5.19   & 1.85 & 0.005 & 25.2 & 6,151 / 8,637 \\
\textbf{p=2, Epanechnikov} & Wage & Population & -229.13 & 37.17 & $<0.001$ & 150.9 & 10,130 / 4,658 \\
                           & Wage & Time       & 336.95  & 18.84 & $<0.001$ & 28.6  & 6,151 / 8,637 \\
                           & Score & Population & -6.91  & 2.98 & 0.020 & 191.2 & 10,130 / 4,658 \\
                           & Score & Time       & 1.08   & 2.88 & 0.709 & 23.6 & 6,151 / 8,637 \\
\hline
\multicolumn{8}{l}{\footnotesize Notes: Local linear regressions estimated with \texttt{rdrobust}. Bandwidths are MSE-optimal.}\\
\multicolumn{8}{l}{\footnotesize Triangular and Epanechnikov kernels used with varying polynomial orders $p \in \{1,2,3\}$.}\\
\multicolumn{8}{l}{\footnotesize Significance: *** $p<0.01$, ** $p<0.05$, * $p<0.1$.}\\
\end{tabular}
\end{table}



\section{3}


\begin{table}[H]
\centering
\caption{Joint Regression Discontinuity using Distance to Rurality Frontier}
\label{tab:rd_rurality}
\begin{tabular}{lcccccc}
\hline
Outcome & RD Estimate ($\hat{\tau}$) & Std. Error & p-value & Bandwidth ($h$) & $N_L$ / $N_R$ & Significance \\
\hline
Wage  & -132.72 & 11.45 & $<0.001$ & 294.85 & 12,192 / 2,596 & *** \\
Score & -7.75   & 1.31  & $<0.001$ & 710.48 & 12,192 / 2,596 & *** \\
\hline
\multicolumn{7}{l}{\footnotesize Notes: Running variable is the signed Euclidean distance to the rurality frontier, }\\
\multicolumn{7}{l}{\footnotesize defined as $d_i = \sqrt{(\text{population}_i - 500)^2 + (\text{time}_i - 120)^2}$, with negative values for treated units.}\\
\multicolumn{7}{l}{\footnotesize Local linear estimates computed using \texttt{rdrobust} with triangular kernel and MSE-optimal bandwidths.}\\
\multicolumn{7}{l}{\footnotesize Significance: *** $p<0.01$, ** $p<0.05$, * $p<0.1$.}\\
\end{tabular}
\end{table}


\section{4}

Correct but pending to add the correct graph!!!



\end{document}