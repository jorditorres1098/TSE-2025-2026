\documentclass{article}
\usepackage{amsmath} 
\usepackage{amsfonts}
\usepackage{booktabs}
\usepackage[a4paper, margin=2.5cm]{geometry}
\usepackage{float}   % for [H]
\usepackage{graphicx}   % for \includegraphics
\usepackage{tabularx}
\usepackage[utf8]{inputenc}
\usepackage{geometry}
\usepackage{booktabs}
\usepackage{longtable}
\usepackage{blindtext}
\usepackage{hyperref}
\usepackage{natbib} % <-- NEW: to handle references
\usepackage{setspace}
\usepackage{array}
\usepackage{dcolumn}
\usepackage{threeparttable}
\usepackage{tikz}
\usepackage{amsmath}
\usetikzlibrary{decorations.pathreplacing}
\usepackage{pdflscape} % in your preamble
\usepackage{tabularray}
\setcounter{secnumdepth}{2}

\setlength\parindent{0pt}




\begin{document}

\title{Homework 2: set identification}
\author{Jordi Torres}
\date{\today}


\maketitle


\subsection*{Question 1}
We can easily show that the identified set is a curve by combining the first two moment equalities:
\[
P_{01}+P_{10}= -\alpha_1 -\alpha_2- \alpha_1\alpha_2.
\]
We also know that in our data $\hat{P}_{01}=0.35$ and $\hat{P}_{10}=0.15$, so that
\[
0.5= -\alpha_1 -\alpha_2- \alpha_1\alpha_2.
\]
Then we can define the identified set $\Theta_I$ as
\[
\Theta_I= \left\{(\alpha_1, \alpha_2)\in [-1,0]^{2}: 0.5=-\alpha_1 -\alpha_2- \alpha_1\alpha_2,\ \exists\,u\in[0,1]\right\}.
\]
To see that this defines a curve in $\mathbb{R}^2$, we can solve explicitly for $\alpha_2$:
\[
\alpha_2 = -\frac{0.5+\alpha_1}{1+\alpha_1}, \qquad \alpha_1\neq -1.
\]
Hence the solution set is the graph of a single-valued function, i.e.\ a one-dimensional curve\footnote{I think we can use Implicit Function Theorem but this should be enough}.

\subsection*{Question 2}

\subsubsection*{a)}
We can create bounds on the observed probabilities and then define moment inequalities. For $y=(1,0)$ we have
\[
\hat P_{01} = -\alpha_1(1+\alpha_2) + u\,(\alpha_1\alpha_2), \qquad u\in[0,1].
\]
Since $\alpha_1,\alpha_2\in[-1,0]$, we have $\alpha_1\alpha_2\ge 0$, so $\hat P_{01}$ is (weakly) increasing in $u$. Hence
\[
-\alpha_1(1+\alpha_2) \;\le\; \hat P_{01} \;\le\; -\alpha_1(1+\alpha_2) + \alpha_1\alpha_2,
\]
which yields the two moment inequalities
\[
-\alpha_1(1+\alpha_2) - \hat P_{01} \;\le\; 0,
\qquad
\hat P_{01} + \alpha_1(1+\alpha_2) - \alpha_1\alpha_2 \;\le\; 0.
\]

Analogously, for $y=(0,1)$:
\[
\hat P_{10} = -\alpha_2(1+\alpha_1) + (1-u)\,(\alpha_1\alpha_2),
\]
which is (weakly) decreasing in $u$, giving
\[
-\alpha_2(1+\alpha_1) \;\le\; \hat P_{10} \;\le\; -\alpha_2(1+\alpha_1) + \alpha_1\alpha_2,
\]
and the two inequalities
\[
-\alpha_2(1+\alpha_1) - \hat P_{10} \;\le\; 0,
\qquad
\hat P_{10} + \alpha_2(1+\alpha_1) - \alpha_1\alpha_2 \;\le\; 0.
\]

Finally, $P_{11}=(1+\alpha_1)(1+\alpha_2)$ provides the equality
\[
(1+\alpha_1)(1+\alpha_2) - \hat P_{11} = 0.
\]

Thus we have a system of 5 (in)equalities that we can use to estimate the bounds of the identified set. 


\subsubsection*{b)}

Will use CHT and implement it , I think


\subsubsection*{c)}
I can do subsampling. Maybe time consuming but should be straightforward. 


\subsubsection*{d)}
Report the results here\dots


\subsection*{e)}
Cox and Shi, to read and then do!

\end{document}