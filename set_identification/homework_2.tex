\documentclass{article}
\usepackage{amsmath} 
\usepackage{amsfonts}
\usepackage{booktabs}
\usepackage[a4paper, margin=2.5cm]{geometry}
\usepackage{float}   % for [H]
\usepackage{graphicx}   % for \includegraphics
\usepackage{tabularx}
\usepackage[utf8]{inputenc}
\usepackage{geometry}
\usepackage{booktabs}
\usepackage{longtable}
\usepackage{blindtext}
\usepackage{hyperref}
\usepackage{natbib} % <-- NEW: to handle references
\usepackage{setspace}
\usepackage{array}
\usepackage{dcolumn}
\usepackage{threeparttable}
\usepackage{tikz}
\usepackage{amsmath}
\usetikzlibrary{decorations.pathreplacing}
\usepackage{pdflscape} % in your preamble
\usepackage{tabularray}
\setcounter{secnumdepth}{2}

\setlength\parindent{0pt}




\begin{document}

\title{Homework 2: set identification}
\author{Jordi Torres}
\date{\today}


\maketitle


\subsection*{Question 1}
We know from the model
\[
\begin{aligned}
P^0_{01} &= -\alpha_1(1+\alpha_2) + u\,\alpha_1\alpha_2,\\
P^0_{10} &= -\alpha_2(1+\alpha_1) + (1-u)\,\alpha_1\alpha_2,\\
P^0_{11} &= (1+\alpha_1)(1+\alpha_2),
\end{aligned}
\qquad
(\alpha_1,\alpha_2)\in[-1,0]^2,\; u\in[0,1].
\]

The third equation does not depend on $u$, so any feasible $(\alpha_1,\alpha_2)$ must satisfy
\[
(1+\alpha_1)(1+\alpha_2) = P^0_{11}.
\]
This immediately implies $1+\alpha_i>0$, so in fact $\alpha_i>-1$ and the boundary $\alpha_1=-1$ is not allowed.

Solving explicitly,
\[
\alpha_2 = -1 + \frac{P^0_{11}}{1+\alpha_1}, \qquad \alpha_1\in(-1,0].
\]
So all admissible parameters lie on this one-dimensional curve.

Now I need to check feasibility of $u\in[0,1]$. From the first equation,
\[
u(\alpha_1,\alpha_2) = \frac{P^0_{01} + \alpha_1(1+\alpha_2)}{\alpha_1\alpha_2},
\qquad\text{so I need } 0 \le u(\alpha_1,\alpha_2) \le 1.
\]
Equivalently,
\[
-\alpha_1(1+\alpha_2) \;\le\; P^0_{01} \;\le\; -\alpha_1(1+\alpha_2)+\alpha_1\alpha_2.
\]

\bigskip

\textbf{Final Identified Set}
\[
\Theta_I = \left\{
(\alpha_1,\alpha_2)\in(-1,0]^2 :
(1+\alpha_1)(1+\alpha_2)=P^0_{11},\;
\frac{P^0_{01} + \alpha_1(1+\alpha_2)}{\alpha_1\alpha_2}\in[0,1]
\right\}.
\]
As expected, it is a curve in $\mathbb{R}^2$, not an area.


\subsection*{Question 2}

\subsubsection*{a)}
We can create boxunds on the observed probabilities and then define moment inequalities. For $y=(1,0)$ we have
\[
\hat P_{01} = -\alpha_1(1+\alpha_2) + u\,(\alpha_1\alpha_2), \qquad u\in[0,1].
\]
Since $\alpha_1,\alpha_2\in[-1,0]$, we have $\alpha_1\alpha_2\ge 0$, so $\hat P_{01}$ is (weakly) increasing in $u$. Hence
\[
-\alpha_1(1+\alpha_2) \;\le\; \hat P_{01} \;\le\; -\alpha_1(1+\alpha_2) + \alpha_1\alpha_2,
\]
which yields the two moment inequalities
\[
-\alpha_1(1+\alpha_2) - \hat P_{01} \;\le\; 0,
\qquad
\hat P_{01} + \alpha_1 \;\le\; 0.
\]

Analogously, for $y=(0,1)$:
\[
\hat P_{10} = -\alpha_2(1+\alpha_1) + (1-u)\,(\alpha_1\alpha_2),
\]
which is (weakly) decreasing in $u$, giving
\[
-\alpha_2(1+\alpha_1) \;\le\; \hat P_{10} \;\le\; -\alpha_2(1+\alpha_1) + \alpha_1\alpha_2,
\]
and the two inequalities
\[
-\alpha_2(1+\alpha_1) - \hat P_{10} \;\le\; 0,
\qquad
\hat P_{10} + \alpha_2 \;\le\; 0.
\]

Finally, $P_{11}=(1+\alpha_1)(1+\alpha_2)$ provides the equality
\[
(1+\alpha_1)(1+\alpha_2) - \hat P_{11} = 0.
\]

Thus, in principle we obtain five (in)equalities. However, in the specific data configuration 
$\hat P_{01}=0.35$, $\hat P_{10}=0.15$, and $\hat P_{11}=0.5$, the system reduces to the simpler set
\[
\hat P_{01} \le -\alpha_1,\qquad \hat P_{10} \le -\alpha_2,\qquad (1+\alpha_1)(1+\alpha_2)=\hat P_{11},
\]
since the lower bounds are automatically satisfied once the upper bounds and the equality are imposed.


\subsubsection*{b)}
I propose the following test:

\[
Tn(\theta)= max_{j} \sqrt{n} \frac{\bar{m_j}}{\sigma_j}
\]


\subsubsection*{c)}
I will compute the critical value using GMS. The idea is to simulate a multivariate normal, that has the variance-covariance matrix of my observed data. Then I define

\[
\xi_{j}(\theta)=\sqrt{n} \frac{\bar{m_j}}{\sigma_j} \frac{1}{\kappa}
\]

Where \(\kappa=\sqrt{2 \log(\log(n))}\). For every value of $\theta$ in the grid (or potential combination) I then compute this variable and add the initially simulated. Then I take the row maximum to have an asymptotic distribution of the test statistic and then I take the 95th percentile to define the critical value.



\subsubsection*{d)}
My bounds are the following: 

\begin{table}[H]
\centering
\caption{GMS 95\% confidence bounds for $(\alpha_1,\alpha_2)$}
\label{tab:gms_bounds}
\begin{tabular}{lcc}
\toprule
Parameter & Lower & Upper \\
\midrule
$\alpha_1$ & $-0.470$ & $-0.315$ \\
$\alpha_2$ & $-0.325$ & $-0.125$ \\
\bottomrule
\end{tabular}
\end{table}

The total number of points are 504.


\subsection*{e)}

For each parameter value $\theta=(\alpha_1,\alpha_2)$ I do the following:

\begin{enumerate}
    \item Compute the sample moment vector
    \[
    m_n(\theta)=
    \begin{pmatrix}
    \hat P_{01} + \alpha_1\\
    \hat P_{10} + \alpha_2\\
    \hat P_{11} - (1+\alpha_1)(1+\alpha_2)
    \end{pmatrix},
    \]
    where the first two are inequality moments ($\le 0$) and the last one is an equality ($=0$).

    \item Estimate the covariance matrix $\hat\Sigma$ of $(\hat P_{01},\hat P_{10},\hat P_{11})$ using the usual multinomial formula
    \[
    \hat\Sigma = \mathrm{diag}(\hat p) - \hat p\,\hat p', \qquad \hat p=(\hat P_{01},\hat P_{10},\hat P_{11})'.
    \]
    This matrix is singular, so I use the Moore--Penrose inverse $\hat\Sigma^+$.

    \item Project $m_n(\theta)$ onto the feasible set
    \[
    \mathcal{M} = \{\mu\in\mathbb{R}^3 : \mu_1\le 0,\ \mu_2\le 0,\ \mu_3=0\}.
    \]
    In other words, I solve numerically
    \[
    \hat\mu(\theta) = \arg\min_{\mu\in\mathcal{M}} \;
    n\,(m_n(\theta)-\mu)'\,\hat\Sigma^+\,(m_n(\theta)-\mu).
    \]
    (In code this is just an \texttt{optim()} with box constraints.)

    \item Count how many moments are ``close to'' binding at the solution $\hat\mu(\theta)$. The equality is always binding, and I add one more for each inequality with $\hat\mu_j\approx 0$.

    \item Compute the critical value as $\chi^2_{r,\,1-\alpha}$ where $r$ is the number of binding moments.

    \item Keep $\theta$ if
    \[
    T_n(\theta)
    :=
    n\,(m_n(\theta)-\hat\mu(\theta))'\,\hat\Sigma^+\,(m_n(\theta)-\hat\mu(\theta))
    \;\le\;
    \chi^2_{r,\,1-\alpha}.
    \]
\end{enumerate}

The confidence set is just all grid points that survive this test. However, there seems to be an error in my code, as I am not able to generate correct bounds.


\end{document}