\documentclass{beamer}

\usetheme{AnnArbor} % AnnArbor main theme
\usecolortheme{dolphin} % Blue color theme

% Sidebar navigation similar to Frankfurt
\useoutertheme[subsection=false]{miniframes}

% Packages
\usepackage{graphicx, booktabs, amsmath, amssymb, hyperref, xcolor}
% Title information
\title[CMT (2021)]{Market Structure and Competition in Airilines Markets}
\author[Zafar, Wiswall]{JPE, 2021 \\Federico Ciliberto, Charles Murry, Elie Tamer}
\date{\today}

\begin{document}

% Title Slide
\begin{frame}
    \titlepage
\end{frame}

% Outline Slide
\begin{frame}{Outline}
    \tableofcontents
\end{frame}

% Section 1: Introduction
\section{Introduction}

\begin{frame}{Introduction}
    \begin{itemize}
        This is a sketch...
        \item Game of simultaneous entry and pricing decisions. 
        \item Let $i=1,2$ denote the firms and $j=1, ..., J$ the number of potential markets 
        \item Then firms need to decide $y_{ij}=\{0,1\}$ and set $P{ij}$ conditional on $y_{ij}=1$.
        \item Problem \rightarrow 
        \begin{enumerate}
            \item Firms select into markets, so entry decision needs to be modelled (vs BLP that takes market entry as given)
            \item This affects the analysis of merger policy and effects.
            \item Basically, expand BLP to introduce endogeneous market entry
        \end{enumerate}
    \end{itemize}
\end{frame}

\begin{frame}{Literature Review}
\begin{enumerate}
    \item BLP: estimate competition model, nash eq, heterogeneous products etc. 
    \item + add entry game very similar as Ciliberto Tamer 2009
    
    Mention other important methods where this has been applied\dots they mention it in their paper. 
    (This is important to have an idea of the key idea of the paper)
\end{enumerate}
    
\end{frame}


\begin{frame}{Simple model}
    full info, pure strategy game (think it can be extended to mixed)
    \begin{equation}
        y_1= \mathbb{1}\left[\delta_{2}y_2+\gamma Z_{1}+ \upsilon_1 \right]
        y_2= \mathbb{1}\left[\delta_{1}y_1+\gamma Z_{2}+ \upsilon_2 \right]

        S_1=X_1\beta + \alpha_1 V_1 + \psi_1
        S_2=X_2\beta + \alpha_2 V_2 + \psi_2
    \end{equation}
    explain the variables that are exogeenous and the endogeneous. The fundamental idea is this one:
    where \(\left(\upsilon_1, upsilon_2, \phi_1, phi_2\right) \sim (N,\Sigma)\)
     
    And the off-diagonal entries of sigma are not 0! this introduces the source of selection biased, explain example here (similar to Heckman's selection model, potentially). 
\end{frame}



\begin{frame}{Simple model}
    \begin{equation}
        y_1= \mathbb{1}\left[\delta_{2}y_2+\gamma Z_{1}+ \upsilon_1 \right]
        y_2= \mathbb{1}\left[\delta_{1}y_1+\gamma Z_{2}+ \upsilon_2 \right]

        S_1=X_1\beta + \alpha_1 V_1 + \psi_1
        S_2=X_2\beta + \alpha_2 V_2 + \psi_2
    \end{equation}
    explain the variables that are exogeenous and the endogeneous. The fundamental idea is this one:
    where \(\left(\upsilon_1, upsilon_2, \phi_1, phi_2\right) \sim (N,\Sigma)\)
     
    And the off-diagonal entries of sigma are not 0! this introduces the source of selection biased, explain example here (similar to Heckman's selection model, potentially). Maybe talk about the non-standard problems that we may have here -multiple eq and setting of price (endogeneity).  
\end{frame}



\begin{frame}{Simple model}
But how to do inference?

We can estimate the distribution of what we observe in the data \(\left(S_1y_1, V_1y_1, y_1, S_2y_2, V_2y_2, y_2\right)\) compare this with simulated moments--> using

\(P(\psi_1\leq t_1; y_1=1 , y2=0 )\) and we can compare this to: 

\(P(\psi_1\leq t_1; y_1=1 , y2=0 )= P(\psi_1\leq t_1; (\upsilon_1, \upsilon_2)\in A^{u}_{(1,0)})+ P(d_{1,0} |\psi_1\leq t_1 ; (\upsilon_1, \upsilon_2)\in A^{nu}_{(1,0)} ) P(\psi_1\leq t_1 ; (\upsilon_1, \upsilon_2)\in A^{nu}_{(1,0)} )\)

So we can easily bound these two inequalities using:

\[
P(\psi_1\leq t_1; (\upsilon_1, \upsilon_2)\in A^{u}_{(1,0)}) \leq P(\psi_1\leq t_1; y_1=1 , y2=0 ) \leq P(\psi_1\leq t_1; (\upsilon_1, \upsilon_2)\in A^{u}_{(1,0)})+ P(\psi_1\leq t_1 ; (\upsilon_1, \upsilon_2)\in A^{nu}_{(1,0)} )
\]

If we repeat this exercise we can repeat it for firm two 0,1 condition. And what is left is 1,1 , 0,0 , but these are easily bounded: 
\(P(y_1=0, y_2=0)=(\upsilon_1\leq -\gamma Z_1,\upsilon_2\leq -\gamma Z_2 )\)
one is the opposite... This gives us these moment conditions


\(\mathbb{E}\left(G(\theta, S1Y1\dots )\right)\leq 0\)

Maybe CT!! on why there are regions of multiplicy and what that means, revisit  this tomorrow. But should be fine.
\end{frame}

\begin{frame}{The structural model}
\begin{equation}
    y_1=1 \iff \pi_1=\left(p_1-c(W_1, \eta_1) M \cdot \tilde{s}_1(\textbf{p},\textbf{X}, \textbf{Y}, \xi)- F(Z_1, \upsilon_1)\right) \geq 0
    y_2=1 \iff \pi_2=\left(p_2-c(W_2, \eta_1) M \cdot \tilde{s}_2(\textbf{p},\textbf{X}, \textbf{Y}, \xi)- F(Z_2, \upsilon_2)\right) \geq 0
    
    S_1= \tilde{s}_1(\textbf{p},\textbf{X}, \textbf{Y}, \xi)
    S_2= \tilde{s}_2(\textbf{p},\textbf{X}, \textbf{Y}, \xi)
    
    \left(p_1-c(W_1, \eta_1)\right) \frac{\partial \tilde{s}_1}{\partial p_1}+\tilde{s}_1(\textbf{p},\textbf{X}, \textbf{Y}, \xi)
    \left(p_2-c(W_2, \eta_2)\right) \frac{\partial \tilde{s}_2}{\partial p_2}+\tilde{s}_2(\textbf{p},\textbf{X}, \textbf{Y}, \xi)

\end{equation}
Mention assumptions on functions-parametrization and assumptions on how unobservables are correlated , which is the essential idea of the paper (accouont for selection problem)
\end{frame}

\begin{frame}
    Describe the algorithm that they use! the main idea is simple and requires to compute first the distribution of errors and then compare them to the simulated probabilities. 
\end{frame}

\begin{frame}
    Actual application
    Show self-selection of carriers
    Show model of BLP vs BLP with endogeneous entry. 

\end{frame}



\end{document}