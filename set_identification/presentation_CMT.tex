\documentclass{beamer}

\usetheme{AnnArbor} % AnnArbor main theme
\usecolortheme{dolphin} % Blue color theme

% Sidebar navigation similar to Frankfurt
\useoutertheme[subsection=false]{miniframes}

% Packages
\usepackage{graphicx, booktabs, amsmath, amssymb, hyperref, xcolor}
\usepackage{amsmath}
\usepackage{amssymb}
\usepackage{amsfonts}
\usepackage{graphicx}
% Title information
\title[CMT (2021)]{Market Structure and Competition in Airilines Markets}
\author[Ciliberto, Murray, Tamer]{JPE, 2021 \\Federico Ciliberto, Charles Murry, Elie Tamer}
\date{\today}

\begin{document}

% Title Slide
\begin{frame}
    \titlepage
\end{frame}

% Outline Slide
\begin{frame}{Outline}
    \tableofcontents
\end{frame}

% Section 1: Introduction
\section{Introduction}

\begin{frame}{Introduction}
    \begin{itemize}
        \item \textbf{Setup:} Two firms ($i=1,2$) across markets ($j=1,\dots,J$).
        \item Each firm chooses:
        \begin{itemize}
            \item Entry decision $y_{ij} \in \{0,1\}$ 
            \item Price $P_{ij}$ if $y_{ij}=1$
        \end{itemize}
        \item \textbf{Key issues:} 
        \begin{enumerate}
            \item Firms self-select into markets $\Rightarrow$ entry must be modeled.
            \item Ignoring this biases demand and cost estimates (contrast with BLP).
            \item Crucial for evaluating merger policy and welfare.
        \end{enumerate}
    \end{itemize}
\end{frame}


\begin{frame}{Literature Review}
    \begin{itemize}
        \item \textbf{BLP (1995):} Demand and pricing in differentiated products, assumes exogenous market structure.
        \item \textbf{Ciliberto \& Tamer (2009):} Entry games with multiple equilibria, partial identification.
        \item \textbf{This paper:} Combines BLP with endogenous entry à la CT (2009).
    \end{itemize}
\end{frame}

\section{The Model}
\begin{frame}{Simple Model}
    \begin{equation*}
        \begin{aligned}
        y_1 &= \mathbb{1}\!\left[\delta_{2}y_2 + \gamma Z_{1} + \nu_1 \geq 0 \right] \\
        y_2 &= \mathbb{1}\!\left[\delta_{1}y_1 + \gamma Z_{2} + \nu_2 \geq 0 \right] \\
        S_1 &= X_1\beta + \alpha_1 V_1 + \xi_1 \\
        S_2 &= X_2\beta + \alpha_2 V_2 + \xi_2
        \end{aligned}
    \end{equation*}
    \vspace{0.3cm}
    \begin{itemize}
        \item Errors are jointly normal $\mathcal{N}(0,\Sigma)$.
        \item Off-diagonal entries of $\Sigma \neq 0 \implies$ \textbf{selection bias}.
        \item Issues: multiple equilibria and endogenous variables.
    \end{itemize}
\end{frame}

\begin{frame}{Inference: Setup}
    \small
    \textbf{Observables:} $(S_1 y_1, V_1 y_1, y_1, \; S_2 y_2, V_2 y_2, y_2)$.
    
    \vspace{0.2cm}
    \textbf{Key Idea:} Link the distribution of observables to the model's predictions.
    
    \vspace{0.2cm}
    For $(y_1,y_2)=(1,0)$:
    \[
    P(\xi_1 \leq t_1; y_1=1,y_2=0)
    \]
    This probability can be decomposed into unique ($A^u$) and multiple ($A^m$) equilibrium regions.
    \[
    \resizebox{\linewidth}{!}{P(\xi_1 \leq t_1; (\nu_1,\nu_2)\in A^{u}_{(1,0)}) 
    \;+\;
    P(d_{1,0}=1 \mid \xi_1 \leq t_1, (\nu_1,\nu_2)\in A^{nu}_{(1,0)})
    \cdot P(\xi_1 \leq t_1; (\nu_1,\nu_2)\in A^{nu}_{(1,0)}) 
    }
    \]
    
\end{frame}

\begin{frame}{Multiplicity regions}
    \centering
    % To add the image, place your image file (e.g., my_image.png) in the same folder as your LaTeX file
    % Then, uncomment the line below and change 'my_image.png' to your file name.
    % \includegraphics[width=0.8\textwidth]{my_image.png}
    Here add the picture from CT 2009
\end{frame}

\begin{frame}{Inference: Bounds}
    \small
    \[
    P(\xi_1 \leq t_1; (\nu_1,\nu_2)\in A^{u}_{(1,0)})
    \]
    \[
    \leq P(S_1 - \alpha_1 V_1 - X_1 \beta \leq t_1; y_1=1,y_2=0)
    \]
    \[
    \leq P(\xi_1 \leq t_1; (\nu_1,\nu_2)\in A^{u}_{(1,0)}) + P(\xi_1 \leq t_1; (\nu_1,\nu_2)\in A^{m}_{(1,0)})
    \]
    \vspace{0.2cm}
\end{frame}

\begin{frame}{All Moment Conditions}
    \small
    \textbf{Moment Inequalities:}
    \begin{itemize}
        \item For $(y_1, y_2)=(1,0)$:
        \[
        ml_{(1,0)}\leq \mathbb{E}[ \mathbb{1}_{\{S_1-\dots\leq t_1, y_1=1, y_2=0\}} ] \leq mu_{(1,0)}
        \]
        \item For $(y_1, y_2)=(0,1)$:
        \[
        ml_{(0,1)}\leq \mathbb{E}[ \mathbb{1}_{\{S_2-\dots\leq t_2, y_1=0, y_2=1\}} ] \leq mu_{(0,1)}
        \]
    \end{itemize}
    \vspace{0.2cm}
    \textbf{Moment Equalities:}
    \begin{itemize}
        \item For $(y_1, y_2)=(1,1)$:
        \[
        \mathbb{E}[ \mathbb{1}_{\{S_1-\dots\leq t_1, S_2-\dots\leq t_2, y_1=1, y_2=1\}} ] = m_{(1,1)}
        \]
        \item For $(y_1, y_2)=(0,0)$:
        \[
        \mathbb{E}[ \mathbb{1}_{\{y_1=0, y_2=0\}} ] = m_{(0,0)}
        \]
    \end{itemize}

    \textbf{Overall Moment Conditions:}
    \[
    \mathbb{E}[G(\theta, S_1y_1, S_2y_2, V_1y_1, V_2y_2, y_1, y_2)|Z, X] \leq 0
    \]
\end{frame}

\begin{frame}{The Structural Model}
    \begin{equation}
        \begin{cases}
        y_1=1 \iff \pi_1=\left(p_1-c(W_1, \eta_1) M \cdot \tilde{s}_1(\textbf{p},\textbf{X}, \textbf{Y}, \xi)- F(Z_1, \upsilon_1)\right) \geq 0 \\
        y_2=1 \iff \pi_2=\left(p_2-c(W_2, \eta_2) M \cdot \tilde{s}_2(\textbf{p},\textbf{X}, \textbf{Y}, \xi)- F(Z_2, \upsilon_2)\right) \geq 0 \\
        S_1= \tilde{s}_1(\textbf{p},\textbf{X}, \textbf{Y}, \xi) \\
        S_2= \tilde{s}_2(\textbf{p},\textbf{X}, \textbf{Y}, \xi) \\
        \left(p_1-c(W_1, \eta_1)\right) \frac{\partial \tilde{s}_1}{\partial p_1}+\tilde{s}_1(\textbf{p},\textbf{X}, \textbf{Y}, \xi) = 0 \\
        \left(p_2-c(W_2, \eta_2)\right) \frac{\partial \tilde{s}_2}{\partial p_2}+\tilde{s}_2(\textbf{p},\textbf{X}, \textbf{Y}, \xi)=0
        \end{cases}
    \end{equation}
\end{frame}

\begin{frame}{Estimation Algorithm}
    \begin{itemize}
        \item Find $\Theta = (\alpha, \beta, \varphi, \gamma, \Sigma)$ that minimizes distance between empirical and simulated distributions.
        \item Candidate parameter value: $\Theta_0$
    \end{itemize}
    \vspace{0.2cm}
    \textbf{Step 1: Empirical CDF}
    \begin{itemize}
        \item Compute residuals from data using $\Theta_0$:
        \[\hat{\xi}, \hat{\eta}\]
        \item Construct empirical CDFs:
        \[
        \hat{P}(\hat{\xi} \leq t_D, \hat{\eta} \leq t_S | \textbf{X}, \textbf{W}, \textbf{Z})
        \]
    \end{itemize}
    \vspace{0.2cm}
    \textbf{Step 2: Simulated CDFs (Bounds)}
    \begin{itemize}
        \item Simulate $(\nu^r, \xi^r, \eta^r)$ from $\mathcal{N}(0, \Sigma_0)$.
        \item For each draw and each of the $2^K-1$ potential market structures:
        \begin{itemize}
            \item Solve the subsystem of demand and FOCs for equilibrium prices and shares, $(\bar{p}^r, \bar{s}^r)$.
            \item Compute profits to determine equilibria.
        \end{itemize}
        \item Construct simulated CDFs for unique ($A^U$) and multiple ($A^M$) equilibrium regions.
        \[
        \hat{P}(\xi^r \leq t_D, \eta^r \leq t_S; \nu^r \in A^U) \quad \text{and} \quad \hat{P}(\xi^r \leq t_D, \eta^r \leq t_S; \nu^r \in A^M)
        \]
    \end{itemize}
    \vspace{0.2cm}
    \textbf{Step 3: Minimum Distance}
    \begin{itemize}
        \item Use a criterion function to define bounds based on how far the simulated moments depart from the moment conditions.
        \item The identified set is the set of all $\Theta$ that minimize this distance.-->CHT 2007
    \end{itemize}
\end{frame}

\begin{frame}{Data}
  %show picture of the results 

\end{frame}

\begin{table}[h!]
    \centering
    \begin{tabular}{lccc}
        \toprule
        \textbf{Demand} & \textbf{GMM} & \textbf{Ex.E} & \textbf{En.E} \\
        \midrule
        Price (100\$) & [-2.385, -2.185] & [-2.315, -2.282] & [-1.557, -1.488] \\
        \midrule
        \textbf{Market Power} & & & \\
        \midrule
        Median Elasticity & [-8.163, -8.091] & [-7.281, -7.063] & [-4.105, -4.007] \\
        Median Markup & [28.146, 28.274] & \textcolor{red}{[30.366, 31.564]} & \textcolor{blue}{[53.617, 56.051]} \\
        \bottomrule
    \end{tabular}
\end{table}


\end{document}