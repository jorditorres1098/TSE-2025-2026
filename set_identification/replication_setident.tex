\documentclass{article}
\usepackage{amsmath} 
\usepackage{amsfonts}
\usepackage{booktabs}
\usepackage[a4paper, margin=2.5cm]{geometry}
\usepackage{float}   % for [H]
\usepackage{graphicx}   % for \includegraphics
\usepackage{tabularx}
\usepackage[utf8]{inputenc}
\usepackage{geometry}
\usepackage{booktabs}
\usepackage{longtable}
\usepackage{blindtext}
\usepackage{hyperref}
\usepackage[round]{natbib}
\usepackage{setspace}
\usepackage{array}
\usepackage{dcolumn}
\usepackage{threeparttable}
\usepackage{tikz}
\usepackage{amsmath}
\usetikzlibrary{decorations.pathreplacing}
\usepackage{pdflscape} % in your preamble
\usepackage{tabularray}
\setcounter{secnumdepth}{2}
\usepackage{amsthm}
\newtheorem{definition}{Definition}



\setlength\parindent{0pt}

\begin{document}

\title{Replication Exercise -Set Identification, Mres}
\author{Jordi Torres}
\date{\today}


\maketitle

\section{Introduction}

This exercise replicates \citet{MHM} - denoted MHM throughout. In that paper, the authors propose a framework to test the Roy selection model using partial identification techniques. They also show how to construct a measure of departure from the Roy model and use inference methods from \citet{CHK} -herbeby denoted CLR-. MHM apply their methodology to German and Canadian data, focusing on selection into STEM versus non-STEM higher education fields and on how this selection affects labor market outcomes such as wages and job finding. Their approach allows them both to test whether the Roy model is rejected and to quantify the magnitude of its failure across regions of observables. \\

In my replication, I apply the same methodology to the Belgian context. Rather than focusing on sorting into STEM fields in higher education, I study selection into STEM versus non-STEM tracks at the high school level, and I use outcomes in higher education, in particular the probability of graduating in STEM or non-STEM degrees, as the relevant outcome of interest. The main goal is to reproduce a table close to Table 5 in MHM (page 3264), which reports estimates of the bounds of the their measure of departure from Roy. This exercise serves both as a replication of their methodology and as an application to a different institutional setting.\\

The remainder of the document is organized as follows. First, in section~\ref{section1} I describe the model setting, following MHM, and I explain how the measure of departure from the Roy model is constructed. Second, in section~\ref{section2} I briefly describe the dataset and the instruments I use in my setting. Third, in \ref{section3} I explain how I do inference on this measure based on the methods proposed by CLR. Finally, in \ref{section4} I present and interpret the empirical results I find and I describe some limitations of this exercise that I would like to improve upon in the future.


\section{Setting}
\label{section1}

Let $Y=\{1,0\}$, where $Y=1$ when the individual in my sample has graduated from STEM degree in university and 0 if the student has graduated in non-STEM degree. Let $D=\{1,0\}$ where $D=1$ if the individual chooses a STEM track in high-school and $D=0$ if the individual chooses a non-STEM track in high-school. Finally, let $Z$ denote a set of instruments.

We will make the following assumptions, following MHM: 

\begin{itemize}
    \item \textbf{Assumption 1} (potential outcomes): $Y=Y_{1}D+ Y_0(1-D)$, where $(Y_{1}, Y_{0})$ is the distribution of the potential outcomes -which we don't observe- and $Y,D$ are what we observe in the data.
    \item \textbf{Assumption 2} (selection): if $Y_{1}>Y_{0} \implies D=1$
    \item \textbf{Assumption 3} (expected outcomes): \(\mathbf{E}(Y_{1}-Y_{0}|\mathcal{I}_{t})>0 \implies D=1\), where $\mathcal{I}_{t}$ denotes the information set that the individual has at time t. 
    \item \textbf{Assumption 4} (SMIV) for any pair $z_{2}\geq z_{1}$ in the support of Z, the conditional distribution of $(Y_{1},Y_{0})$ given $Z=z_{2}$ first order stochastically dominates the distribution of $(Y_{1},Y_{0})$ given $Z=z_{1}$ \footnote{This is very similar to the Manski and Pepper (2000) MIV with the difference that this assumption is stronger. M-P assumed monotonicity of expected potential outcome , i.e $\mathbf{E}(Y_1|Z=z)$, but the assumption that we are doing is in the joint distribution. What we are looking for is instruments that affect both potential outcomes at the same time}. 
\end{itemize}

Under ASSUMPTIONS 1,3,4, MHM show that we can construct the following bounds on the joint distribution of potential outcomes: 


\begin{enumerate}
    \item \(\mathbf{P}(Y_{0}=1, Y_{1}=0 |Z=z)\leq \mathbf{P}(Y=1, D=0 |Z=z)+ \mathbf{P}(Y=0, D=1 |Z=z)\)
    \item \(\mathbf{P}(Y_{0}=0, Y_{1}=1 |Z=z)\leq \mathbf{P}(Y=1, D=1 |Z=z)+ \mathbf{P}(Y=0, D=0 |Z=z)\)
    \item \(\mathbf{P}(Y_{0}=0, Y_{1}=0 |Z=z)\leq \mathbf{P}(Y=0|Z=z)\) \\


    And, from the joint distribution, combining 1,2 and 3 we can derive conditions for the marginal distribution of outcomes: \\

    \item  \(\sup_{\tilde{z}\leq z}\mathbf{P}(Y=1, D=0|Z=\tilde{z})\leq \mathbf{E}(Y_{0}| Z=z)\)
    \item  \(\sup_{\tilde{z}\leq z}\mathbf{P}(Y=1, D=1|Z=\tilde{z}) \leq \mathbf{E}(Y_{1}|Z=z)\)
    \item \(\max\{\mathbf{E}(Y_{0}|Z=z), \mathbf{E}(Y_{1}|Z=z)\}= \mathbf{E}(y|Z=z)\)
\end{enumerate}

These six inequalities define the set of joint distributions of potential outcomes that are consistent with the model derived under the 3 previous assumptions. In the paper they show that we can test the validity of the assumptions by simply testing that the last equality holds (\textbf{Theorem 2})  which then simply boils down to testing monothonicity of $\mathbf{E}(Y|Z=z)$ in Z. This can be done by using methods such as \citet{chetverikov} and \citet{Hsu_Liu_Shi_2019}. \\

Although in my code I have tried to test for monothonicity too\footnote{For example, in the appendix of the Jupyter Notebook, I have tried to test for monothonicity of $\mathbf{E}(Y|Z=z)$ with z discrete by simply setting the inequalities and using Andrews and Shi -and the methodology used in class-, simulating the critical values by bootstrap. Nevertheless, the test was too restrictive and almost never I rejected that Roy model was consistent with the data.} I ran into complications and decided that the main effort of my exercise would be to focus in replicating el(z) measure and in doing inference based on CLR methods. This is the next step: \\


\textbf{Measure of departure from Roy} \\

The second contribution of the paper is to generate a measure of rejection of Roy model behavior, which they call efficiency loss. This measure determines the extend towards which there are other factors influencing sorting besides potential outcomes (e.g in our setting, for example, the extend to which some girls may not be sorting into stem fields for which they have a comparative advantage, maybe because of social rules etc). \\

\begin{definition}
    \textbf{Efficiency loss.} For each $z$ in the support of $Z$, they define
    \[
        el(z)
        := \mathbb{P}(\max\{Y_0,Y_1\}=1 \mid Z=z)
         - \mathbb{P}(Y=1 \mid Z=z).
    \]
\end{definition}

In the binary case $Y_d \in \{0,1\}$,then $\max\{Y_0,Y_1\}=1$ except when $(Y_0,Y_1)=(0,0)$. Therefore:
\[
    \mathbb{P}(\max\{Y_0,Y_1\}=1 \mid Z=z)
    = 1 - \mathbb{P}(Y_0=0,Y_1=0 \mid Z=z),
\]
and we can also write
\[
    el(z)
    = \mathbb{P}(Y=0 \mid Z=z) - \mathbb{P}(Y_0=0,Y_1=0 \mid Z=z).
\]
By construction, $el(z) \ge 0$ for all $z$, since $\max\{Y_0,Y_1\} \ge Y$. \\

\paragraph{Perfect foresight Roy.}
Under Assumption~2 (perfect foresight Roy selection), individuals always choose the track that yields the highest realization of $Y_d$, so
\[
    Y = \max\{Y_0,Y_1\}.
\]
It follows that
\[
    \mathbb{P}(Y=0 \mid Z=z)
    = \mathbb{P}(\max\{Y_0,Y_1\}=0 \mid Z=z)
    = \mathbb{P}(Y_0=0,Y_1=0 \mid Z=z),
\]
and therefore
\[
    el(z) = 0 \quad \text{for all } z.
\]
Thus, any positive efficiency loss is evidence against perfect foresight. Note that under assumptions 1,3,4  -imperfect foresight only- this measure is again only partially identified and can be different from 0.


\paragraph{SMIV without Roy selection.}
If we drop the Roy selection rule altogether and only maintain Assumptions~1 and 4, SMIV alone still imposes stochastic monotonicity restrictions on $(Y_0,Y_1)$ across values of $Z$. These imply a lower bound on:
\[
    L_{\mathrm{SMIV}}(z)
    := \inf_{\tilde{z} \le z} \mathbb{P}(Y=0 \mid Z=\tilde{z})
    \;\le\;
    \mathbb{P}(Y_0=0,Y_1=0 \mid Z=z),
\]
together with the trivial upper bound $\mathbb{P}(Y_0=0,Y_1=0 \mid Z=z) \le \mathbb{P}(Y=0 \mid Z=z)$. Consequently,
\[
    \mathbb{P}(Y=0 \mid Z=z) - L_{\mathrm{SMIV}}(z) \le el(z)
    .
\]


\textbf{In my replication exercise I have focused in estimating $el(z)$ under assumptions 1-4 only}, as this is the key measure of MHM and the one in which they apply CLR inference methods.



\section{Sample}

\label{section2}
The empirical analysis uses a sample of 2,181 belgian students for whom I observe their complete high-school trajectory and up to two years of higher-education. For each individual, I observe whether the student chooses a STEM or non-STEM track in the beginning of high school and the type of degree the students enrolls in in university. Given that I don't observe the final graduation, I assume that if students are in year 2 in a STEM degree they will eventually graduate and take this as the outcome of interest\footnote{In the Belgian context this assumption, although strong, is defensible. In Belgium, there are no entry requirements in universities in terms of grades or background. A student that comes from non-STEM background can enter an Engineering degree. However, because of this, universities impose strong academic requiremenents in year 1 to filter out students with low ability. Having survived year 1 is therefore a good predictor of eventual graduation.} In addition, the dataset contains rich background information, including parental income, parental education, IQ measures, gender, and other individual characteristics. \\

To explore heterogeneity in the measure of departure from Roy, I construct 10 subsamples of the original dataset by binning individuals according to gender and 5 family income groups. The objective is to examine how the efficiency loss $el(z)$ varies across these dimensions. This is very similar to what MHM do (they don't observe income, but bin in terms of minority groups). \\

Finally I consider two instruments $Z$. The first is parental education, measured as a discrete variable with three categories, which is also used as an instrument in MHM. The second is the student’s IQ score, which was measured before students got into high-school in a standardized exam. I assume that IQ affects both potential outcomes positively $Y_1$ and $Y_0$ and therefore satisfies the stochastic monotonicity assumption. Using both a discrete and a continuous instrument allows me to implement the CLR inference procedure in both settings and to assess the sensitivity of the results to the nature of the instrument.


\section{Inference: Bounds on efficiency measure}
\label{section3}

I implement the intersection-bounds inference procedure, following closely the algorithm proposed by \citet{CHK} (Section 6.1). Note that in my case I am trying to derive bounds on the functional $el(z)$, which is bounded by below by $\mathbb{P}(Y=0 \mid Z=z) - L_{\mathrm{SMIV}}(z)$. This depends on Z that, in the most general case I have, can be continuous.  The procedure I describe here is applied separately to each subsample\footnote{In the Julia code all the functions are clearly explained and detailed.}:

\begin{enumerate}
\item \textbf{Series estimation of $\mathbb{P}(Y=0\mid Z=z)$.}  
I estimate the conditional probability $\theta(z)=\mathbb{P}(Y=0\mid Z=z)$ by a series regression of $\mathbf{1}\{Y=0\}$ on a polynomial basis in $Z$. Let
\[
p(Z_i)=(1,Z_i,Z_i^2,\dots,Z_i^{K-1})',
\]
and let $P$ be the $n\times K$ matrix stacking $p(Z_i)'$. The coefficient vector is estimated by OLS,
\[
\hat\beta_n=(P'P)^{-1}P'Y^{(0)},
\]
where $Y^{(0)}=\mathbf{1}\{Y=0\}$. The fitted bounding function is
\[
\widehat{\theta}_n(z)=p(z)'\hat\beta_n,
\]
which is evaluated on a finite grid of values of $Z$.

\item \textbf{Variance estimation.}  
I compute a consistent estimator $\hat\Omega_n$ of the asymptotic variance of $\sqrt{n}(\hat\beta_n-\beta_n)$ using a standard sandwich formula, and obtain its symmetric square root $\hat\Omega_n^{1/2}$.

\item \textbf{Moment inequalities.}  
For each grid point $z_j$\footnote{I use 100 to approximate the support of Z.}, the lower bound
\[
\inf_{\tilde z\le z_j}\theta(\tilde z)
\]
can be written as a finite collection of inequalities
\[
\theta(z_j)-\theta(z_i)\ge 0,\qquad i<j.
\]
These inequalities define an index set $V_j=\{1,\dots,j-1\}$. For each $v\in V_j$, I define
\[
\hat\theta_n(v)=\widehat{\theta}_n(z_j)-\widehat{\theta}_n(z_v),
\qquad
p_n(v)=p(z_j)-p(z_v).
\]

\item \textbf{Standardized moments.}  
For each $v\in V_j$, I compute
\[
\hat g(v)=p_n(v)'\hat\Omega_n^{1/2},
\qquad
s_n(v)=\frac{\|\hat g(v)\|}{\sqrt{n}}.
\]

\item \textbf{Simulation and first-stage critical value.}  
I set $\tilde\gamma_n=1-0.1/\log n$ and simulate $R$ draws $Z_r\sim\mathcal{N}(0,I_K)$. The first-stage critical value $\kappa_{n,V_j}(\tilde\gamma_n)$ is computed as the $\tilde\gamma_n$-quantile of
\[
\left\{
\sup_{v\in V_j}\frac{\hat g(v)'Z_r}{\|\hat g(v)\|}
: r=1,\dots,R
\right\}.
\]

\item \textbf{Inequality selection.}  
Following CLR, I construct the reduced index set $\hat V_{n,j}$ by retaining inequalities that are close to binding.

\item \textbf{Second-stage critical value and lower bound.}  
Using $\hat V_{n,j}$ in 5, I compute a second-stage critical value $\kappa_{n,\hat V_{n,j}}(p)$ -with $p=0.90$- and form the one-sided lower confidence bound
\[
\hat\theta_{n,0}(p)
=
\inf_{v\in \hat V_{n,j}}
\left[
\hat\theta_n(v)+\kappa_{n,\hat V_{n,j}}(p)s_n(v)
\right].
\]
Repeating this procedure for each grid point gives a lower confidence band for the bounding function over z.\footnote{When $Z$ is discrete, several steps simplify: $\theta(z)$ is estimated by sample means, and the series and variance estimation steps are no longer required and are much simpler.}
\end{enumerate}




\section{Results}
\label{section4}

Figures \ref{fig:eff_loss_women} and \ref{fig:eff_loss_men} report the 90\% lower confidence bounds on the efficiency loss function $el(z)$ by gender, income group, and values of the instrument $Z$, a very similar table as FIG. 5 in MHM. Dark regions correspond to values of $(z,\text{income})$ for which the data are consistent with Roy selection, while lighter regions indicate statistically significant departures, in the sense that the lower confidence bound on $el(z)$ is strictly positive. \\

For women, departures from Roy selection are limited and concentrated primarily in middle and higher income groups. Overall, the magnitude of the departures is very small, with lower confidence bounds rarely exceeding 0.015. For men, violations of Roy selection are larger and more frequent. In particular, positive lower bounds are concentrated in middle income groups and appear both at low and high values of $z$.  \\

In appendix, tables~\ref{fig:eff_loss_women_dis} and \ref{fig:eff_loss_men_dis} I show the results for the discrete instrument (parental education). Here we can see that lower bounds on $el(z)$ are much larger, particularly for high income and middle SES students. Efficiency departures are also high for girls with high SES and for boys in middle SES distribution. \\

Compared to MHM, the qualitative structure of the results is similar, in that departures from Roy selection vary systematically with background characteristics and the instrument. However, the magnitude of the departures is substantially smaller in the Belgian data, indicating more limited deviations from Roy selection than those documented in their application. Also, contrary to their resutls, it seems that girls are consistent with Roy model far more than boys\footnote{Of course, my Roy setting is different, so it is hard to compare both results}. One interpretation that can be rationalized with the results of the model is that boys may overshot and choose stem in high-school when they should not (it is the prestigious track), while girls tend to choose stem tracks in high-school in a very conservative way (only when it is Roy consistent), driven by social-norms. \\

Finally, this project can be benefited from a better testing of the monothonicity assumption. Due to time constraints, I did not manage to implement \citet{Hsu_Liu_Shi_2019}, which is the core method used by MHM to test their model. In the future, I would like to test for this using the full machinery they use.  


\begin{figure}[H]
    \centering
    \includegraphics[width=0.8\textwidth]{efficiency_loss_women.png}
    \caption{Efficiency loss (women), 90\% confidence level}
    \label{fig:eff_loss_women}
\end{figure}



\begin{figure}[H]
    \centering
    \includegraphics[width=0.8\textwidth]{efficiency_loss_men.png}
    \caption{Efficiency loss (men), 90\% confidence level}
    \label{fig:eff_loss_men}
\end{figure}

\section{Appendix}

\begin{figure}[H]
    \centering
    \includegraphics[width=0.8\textwidth]{efficiency_loss_women_discrete.png}
    \caption{Efficiency loss (women), 90\% confidence level}
    \label{fig:eff_loss_women_dis}
\end{figure}


\begin{figure}[H]
    \centering
    \includegraphics[width=0.8\textwidth]{efficiency_loss_men_discrete.png}
    \caption{Efficiency loss (men), 90\% confidence level}
    \label{fig:eff_loss_men_dis}
\end{figure}

\bibliographystyle{apalike}

\bibliography{bibliography}

\end{document}