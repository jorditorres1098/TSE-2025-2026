\documentclass{beamer}
\usetheme{Madrid}

\title{Research Proposal: Socio-Emotional Skills and Experimental Evaluation}
\author{Your Name}
\date{\today}

\begin{document}

\frame{\titlepage}

% --- Slide 1: The Idea
\begin{frame}{The Idea and Why an Experiment}
    \begin{itemize}
        \item Literature (e.g., Heckman): socio-emotional skills (SES) crucial for long-run outcomes.
        \item Schools may foster SES through structured training programs.
        \item Problem: SES usually measured by self-reports, which are noisy and biased.
        \item Solution: Randomized controlled trial + standardized behavioral tasks
            \begin{itemize}
                \item Randomization ensures causal identification.
                \item Observed cooperation and productivity provide objective measures.
            \end{itemize}
    \end{itemize}
\end{frame}

% --- Slide 2: Research Question 1
\begin{frame}{Research Question 1}
    \begin{block}{Main Question}
        Does classroom-based SES training causally increase cooperation and productivity in a standardized team task (e.g., tower-building)?
    \end{block}
    \begin{itemize}
        \item Randomize training across classrooms.
        \item Measure group performance and cooperative behaviors in a controlled setting with given incentives. 
        \item Outcomes: task output, conflict resolution, coordination quality.
    \end{itemize}
\end{frame}

% --- Slide 3: Research Question 2
\begin{frame}{Research Question 2}
    \begin{block}{Main Question}
        Does SES training foster pro-social behavior in economic interactions?
    \end{block}
    \begin{itemize}
        \item Use canonical experimental games: public goods, trust, coordination.
        \item Compare treated vs. control students on contribution, reciprocity, efficiency.
        \item Explore spillovers: do trained peers influence untreated classmates?
    \end{itemize}
\end{frame}

\end{document}
