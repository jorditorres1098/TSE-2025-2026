\documentclass{beamer}
\usetheme{Madrid}

\title{Research Proposal: Socio-Emotional Skills and Experimental Evaluation}
\author{Jordi Torres-Vallverdú}
\date{\today}

\begin{document}

\frame{\titlepage}

% --- Slide 1: The Idea
\begin{frame}{Motivation and Research Idea}
    \begin{itemize}
        \item \textbf{Motivation:} Socio-emotional skills (SES) are crucial for long-run outcomes 
              (education, labor markets, health) \textit{(Heckman et al.)}.
        \vspace{0.3cm}
        \item \textbf{Question 1:} Can SES be improved through school-based interventions?
        \begin{itemize}
            \item Randomize a teacher training program (focus on SES evaluation, feedback, reinforcement).
        \end{itemize}
        \vspace{0.3cm}
        \item \textbf{Question 2:} How can we measure SES reliably?
        \begin{itemize}
            \item Standard self-reports intruments (big5, BESSI...etc) are biased and noisy.
            \item Solution: use standardized contests/games (e.g., tower-building) to observe and measure actual outcomes such as 
                  cooperation, problem-solving skills in action, in a controled environment.
        \end{itemize}
        \vspace{0.3cm}
        \item \textbf{Why experiment?} Randomization ensures causal identification of SES training 
              effects and, more importantly, \textbf{behavioral measures provide objective outcomes.}
    \end{itemize}
\end{frame}


% --- Slide 2: Research Question 1
\begin{frame}{Research Question 1}
    \begin{block}{Main Question}
        Does classroom-based SES training causally increase cooperation and productivity in a standardized team task (e.g., tower-building)?
    \end{block}
    \begin{itemize}
        \item Randomize training across classrooms.
        \item Measure periodically group performance and cooperative behaviors in a controlled setting. 
    \end{itemize}
\end{frame}

% --- Slide 3: Research Question 2
\begin{frame}{Research Question 2}
    \begin{block}{Main Question}
        Does SES training foster pro-social behavior in economic interactions?
    \end{block}
    \begin{itemize}
        \item Use canonical experimental games: public goods, trust, coordination.
        \item The idea is to see if changes in SES also translate into changes in pro-social behavior. 
        \item Addition: explore spillovers? $\rightarrow$ do trained peers influence untreated classmates?
    \end{itemize}
\end{frame}

\end{document}
